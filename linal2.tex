\documentclass[a4paper,12pt,titlepage,final]{article}
% при подготовке финальной версии отчёта смените опцию draft на final

\usepackage[T1,T2A]{fontenc}     % форматы шрифтов
\usepackage[utf8x]{inputenc}     % кодировка символов, используемая в данном файле
\usepackage[russian]{babel}      % пакет русификации
\usepackage{tikz}                % для создания иллюстраций
\usepackage{pgfplots}            % для вывода графиков функций
\usepackage{geometry}		     % для настройки размера полей
\usepackage{indentfirst}         % для отступа в первом абзаце секции
\usepackage{listings}			 % для текста Make-файла
\usepackage{amsmath}			 % для формул
\usepackage{amssymb}			 % для символов

% выбираем размер листа А4, все поля ставим по 3см
\geometry{a4paper,left=30mm,top=30mm,bottom=30mm,right=30mm}
% добавляем точку после номера раздела
\renewcommand{\thesection}{\arabic{section}.}
\usepgfplotslibrary{fillbetween} % для изображения областей на графиках

% автор, дата и название документа
\title{Сборник задач и их решений по предмету "Линейная алгебра", 1 поток, 2 семестр}
\author{ Над этим дерьмом трудились\\Денисов Егор, Долгушев Глеб, Курбацкий Вячеслав из 201 группы}
\date{Когда-то в 2024 году...}
	
\begin{document}

\maketitle

% Оглавление
\newpage
\tableofcontents

% 1 раздел - линейные операторы и жорданки
\newpage
\section*{Линейные операторы и жорданки}
\addcontentsline{toc}{section}{Линейные операторы и жорданки}

% 1 задача
\subsection*{Задача 1:}
\addcontentsline{toc}{subsection}{Задача 1:}
\noindent Докажите, что линейная операция линейных операторов является линейным оператором. \\ \\
% Решение 1 задачи
\textbf{\underline{Решение:}} \\
Пусть $A, B: V \rightarrow W$ - линейные операторы над полем $\mathbb{P}$, a $C = \alpha A + \beta B$ - их линейная комбинация.\\
Тогда $\forall x, y \in V$ и $\forall \delta, \gamma \in \mathbb{P}$:
$C(\delta x + \gamma y) = (\alpha A + \beta B) (\delta x + \gamma y) = \alpha A(\delta x + \gamma y) + \beta B(\delta x + \gamma y) =
\alpha (\delta Ax + \gamma Ay) + \beta (\delta Bx + \gamma By) = \delta (\alpha Ax + \beta Bx) + \gamma (\alpha Ay + \beta By) =
\delta (\alpha A + \beta B)x + \gamma (\alpha A + \beta B)y = \delta Cx + \gamma Cy$ \\ $\blacksquare$ \\ \\ \\

% 2 задача
\subsection*{Задача 2:}
\addcontentsline{toc}{subsection}{Задача 2:}
\noindent Докажите, что произведение линейных операторов является линейным оператором. \\ \\
% Решение 2 задачи
\textbf{\underline{Решение:}} \\
Пусть $A: V \rightarrow W$ и $B: W \rightarrow U$ - линейные операторы над полем $\mathbb{P}$, a $C = B \circ A$ - их произведение.\\
Тогда $\forall x, y \in V$ и $\forall \alpha, \beta \in \mathbb{P}$:
$C(\alpha x + \beta y) = B(A(\alpha x + \beta y)) = B(\alpha Ax + \beta Ay) = \alpha B(Ax) + \beta B(Ay) = \alpha Cx + \beta Cy$ \\ $\blacksquare$ \\ \\ \\

% 3 задача
\subsection*{Задача 3:}
\addcontentsline{toc}{subsection}{Задача 3:}
\noindent Докажите, что если линейный оператор обратим, то обратный оператор определен однозначно и является линейным. \\ \\
% Решение 3 задачи
\textbf{\underline{Решение:}} \\
1. Единственность: \\
Пусть $A: V \rightarrow W$ - обратимый линейный оператор над полем $\mathbb{P}$, a $B$ и $C$ - его обратные операторы.\\
Тогда $\forall x \in W: Bx = B(AC)x = (BA)Cx = Cx \Rightarrow B = C$ \\ \\
2. Линейность: \\
Пусть $A: V \rightarrow W$ - обратимый линейный оператор над полем $\mathbb{P}$, a $B$ - его обратный оператор.\\
$\forall y_1, y_2 \in W$ имеют вид $ y_1 = Ax_1, y_2 = Ax_2, x_1, x_2 \in V \Rightarrow x_1 = By_1, x_2 = By_2$ \\
Тогда $A(\alpha x_1 + \beta x_2) = \alpha (AB)y_1 + \beta (AB)y_2 = \alpha y_1 + \beta y_2 \Rightarrow B(\alpha y_1 + \beta y_2) =$
(так как $B$ - обратный, то из $Ax = y$ следует $By = x$) $= \alpha x_1 + \beta x_2 = \alpha By_1 + \beta By_2$ \\ $\blacksquare$ \\ \\ \\


% 4 задача
\subsection*{Задача 4:}
\addcontentsline{toc}{subsection}{Задача 4:}
\noindent Пусть $V$ и $W$ - конечномерные пространства над общим полем. Докажите, что для обратимости линейного оператора $A: V \rightarrow W$
необходимо и достаточно выполнение условий $dim V = dim W$ и $ker A = \{0\}$ \\ \\
% Решение 4 задачи
\textbf{\underline{Решение:}} \\
1. $\Rightarrow:$ \\
Пусть $A: V \rightarrow W$ - обратимый линейный оператор над полем $\mathbb{P}$ и $B$ - его обратный оператор. \\
Если $kerA \neq \{0\}$, то $\exists x \neq 0: Ax = 0 \Rightarrow x = B(Ax) = B0 = 0 \Rightarrow x = 0$ - противоречие. \\
Так как $AV = W$, то $dimW \leq dimV$. Поясним этот момент - пусть $x_1, x_2, ..., x_n$ - базис $V$.
Тогда $\forall y \in W: y = Ax = A(\alpha_1 x_1 + ... + \alpha_n x_n) = \alpha_1 (Ax_1) + ... + \alpha_n  (Ax_n)$.
То есть все векторы из $W$ выражаются через векторы $Ax_1, ... , Ax_n$. Значит в базисе $W$ векторов не больше, чем $n$.
Аналогично $BW = V \Rightarrow dimV \leq dimW$. Значит $dimV = dimW$. \\ \\
2. $\Leftarrow$: \\
Пусть $dimV = dimW = n$ и $kerA = \{0\}$. Тогда по теореме о ранге и дефекте - $rankA = n$.
Так как $imA \subseteq W$ и $dim(imA) = dimW$, то $imA = W \Rightarrow$ $A$ - сюръективный. \\
Так как $kerA = \{0\}$, то $A$ - инъективный (иначе, если $\exists x_1 \neq x_2: Ax_1 = Ax_2$, то $A(x_1 - x_2) = Ax_3 = 0$, где $x_3 \neq 0 \Rightarrow
kerA \neq \{0\}$ - противоречие). \\
То есть $\forall y \in W$ $\exists ! x \in V: Ax = y \Rightarrow$ построим оператор $B: W \rightarrow V$ таким образом: $By = x$, где $x$ - такой, что $Ax = y$.
Построенный оператор $B$ по определению - обратный для $A$ (так как $(AB)y = y$ и $(BA)x = x \forall x \in V, y \in W$). \\ $\blacksquare$ \\ \\ \\


% 5 задача
\subsection*{Задача 5:}
\addcontentsline{toc}{subsection}{Задача 5:}
\noindent Докажите, что ядро и образ линейного оператора являются его инвариантными подпространствами. \\ \\
% Решение 5 задачи
\textbf{\underline{Решение:}} \\
Пусть $A: V \rightarrow V$ - линейный оператор над полем $\mathbb{P}$. \\
$\forall x \in kerA: Ax = 0 \in kerA \Rightarrow kerA$ - инвариантное подпространство. \\
$\forall y \in imA: Ay \in imA$ по определению образа $\Rightarrow imA$ - инвариантное подпространство. \\ $\blacksquare$ \\ \\ \\


% 6 задача
\subsection*{Задача 6:}
\addcontentsline{toc}{subsection}{Задача 6:}
\noindent Докажите, что сумма ранга и дефекта линейного оператора равна размерности его области определения. \\ \\
% Решение 6 задачи
\textbf{\underline{Решение:}} \\
Пусть $A: V \rightarrow W$ - линейный оператор над полем $\mathbb{P}$. \\
Пусть $defA = d$ и $v_1, ... , v_d$ - базис $kerA$. Дополним его до базиса всего пространства $V$ векторами $v_{d+1}, ... , v_n$. \\
Тогда $\forall x \in V: x = \alpha_1 v_1 + ... + \alpha_d v_d + \alpha_{d+1} v_{d+1} + ... + \alpha_n v_n \Rightarrow
Ax = A(\alpha_1 v_1 + ... + \alpha_d v_d + \alpha_{d+1} v_{d+1} + ... + \alpha_n v_n) =
\alpha_1 Av_1 + ... + \alpha_d Av_d + \alpha_{d+1} Av_{d+1} + ... + \alpha_n Av_n = \alpha_{d+1} Av_{d+1} + ... + \alpha_n Av_n$ (так как $Av_1 = ... = Av_d = 0$) \\
Таким образом, $imA = L(Av_{d+1}, ... , Av_n)$. Докажем, что эти векторы линейно независимы: \\
Пусть $\alpha_{d+1} Av_{d+1} + ... + \alpha_n Av_n = 0 \Rightarrow A(\alpha_{d+1} v_{d+1} + ... + \alpha_n v_n) = 0
\Rightarrow \alpha_{d+1} v_{d+1} + ... + \alpha_n v_n \in kerA \Rightarrow \alpha_{d+1} v_{d+1} + ... + \alpha_n v_n = \alpha_1 v_1 + ... + \alpha_d v_d \Rightarrow
\alpha_1 v_1 + ... + \alpha_d v_d - \alpha_{d+1} v_{d+1} - ... - \alpha_n v_n = 0 \Rightarrow \alpha_{d+1} = ... = \alpha_n = 0$ (так как $v_1, ... , v_n$ - линейно независимы).
То есть из $\alpha_{d+1} Av_{d+1} + ... + \alpha_n Av_n = 0$ следует $\alpha_{d+1} = ... = \alpha_n = 0 \Rightarrow Av_{d+1}, ... , Av_n$ - линейно независимые
$\Rightarrow rankA = n - d \Rightarrow rankA + defA = dimV$. \\ $\blacksquare$ \\ \\ \\


% 7 задача
\subsection*{Задача 7:}
\addcontentsline{toc}{subsection}{Задача 7:}
\noindent Докажите, что если сумма ядер двух линейных операторов, действующих на одном пространстве, совпадает с этим пространством,
то образ суммы этих операторов равен сумме их образов. \\ \\
% Решение 7 задачи
\textbf{\underline{Решение:}} \\
Пусть $A, B: V \rightarrow V$ - линейные операторы над полем $\mathbb{P}$ и $kerA + kerB = V$. \\
Пусть $x \in im(A + B) \Rightarrow \exists y \in V: (A + B)y = x \Rightarrow Ay + By = x \Rightarrow x \in imA + imB
\Rightarrow im(A + B) \subseteq imA + imB$. \\
Пусть $x \in imA + imB \Rightarrow \exists y_1, y_2 \in V: x = Ay_1 + By_2$. Так как $kerA + kerB = V$, то
$y_1 = u_1 + v_1, y_2 = u_2 + v_2$, где $u_1, u_2 \in kerA; v_1, v_2 \in kerB$. Значит $x = A(u_1 + v_1) + B(u_2 + v_2) =
Av_1 + Bu_2 = Av_1 + 0 + Bu_2 + 0 = Av_1 + Au_2 + Bu_2 + Bv_1 = (A + B)(v_1 + u_2) \Rightarrow x \in im(A + B)
\Rightarrow imA + imB \subseteq im(A + B)$.\\
То есть $im(A + B) = imA + imB$. \\ $\blacksquare$ \\ \\ \\


% 8 задача
\subsection*{Задача 8:}
\addcontentsline{toc}{subsection}{Задача 8:}
\noindent Линейный оператор $A: V \rightarrow V$ удовлетворяет равенству $A^m = 0$. Докажите, что оператор
$I - A$ обратим. \\ \\
% Решение 8 задачи
\textbf{\underline{Решение:}} \\
Пусть $A: V \rightarrow V$ - линейный оператор над полем $\mathbb{P}$ и $A^m = 0$. \\
Тогда $(I - A)(I + A + A^2 + ... + A^{m-1}) = (I + A + A^2 + ... + A^{m-1})(I - A) = I - A^m = I \Rightarrow I - A$ - обратим. \\ $\blacksquare$ \\ \\ \\


% 9 задача
\subsection*{Задача 9:}
\addcontentsline{toc}{subsection}{Задача 9:}
\noindent Линейные операторы $A$ и $B$ таковы, что оператор $A + B$ - обратимый. Докажите, что операторы
$P = (A + B)^{-1}A$ и $Q = (A + B)^{-1}B$ коммутируют. \\ \\
% Решение 9 задачи
\textbf{\underline{Решение:}} \\
Заметим, что $P + Q = (A + B)^{-1}(A + B) = I \Rightarrow P = I - Q$. \\
Тогда $PQ = (I - Q)Q = Q - Q^2 = Q(I - Q) = QP$. \\ $\blacksquare$ \\ \\ \\ 


% 10 задача
\subsection*{Задача 10:}
\addcontentsline{toc}{subsection}{Задача 10:}
\noindent Докажите, что для того, чтобы линейный оператор $P: V \rightarrow V$ был оператором проектирования,
необходимо и достаточно, чтобы $P^2 = P$. \\ \\
% Решение 10 задачи
\textbf{\underline{Решение:}} \\
1. $\Rightarrow:$ \\
Пусть $V = L \oplus M$ и $P$ - оператор проектирования на $L$ параллельно $M$, то есть
$\forall x \in V: x = u + v; u \in L, v \in M$ имеем $Px = u$. \\
Тогда $\forall x \in V: x = u + v$ выполняется $P^2x = P(Px) = P(u) = u = Px \Rightarrow P^2 = P$. \\ \\
2. $\Leftarrow:$ \\
Пусть $P^2 = P$. Тогда $V = L \oplus M, где L = imP, M = kerP$. Докажем это: \\
\begin {itemize}
\item По теореме о ранге и дефекте знаем, что $dim(imP) + dim(kerP) = rankP + defP = dim V$ \\
\item Если $x \in kerP \cap imP$, то $Px = 0$ и $x = Py \Rightarrow Px = P^2y = Py = x = 0 \Rightarrow kerP \cap imP = \{0\}$\\
\item Тогда, если $x_1, ... , x_d$ - базис $kerP$, а $x_{d+1}, ... , x_n$ - базис $imP$, то
$x_1, ... , x_n$ все вместе линейно независимы (так как если $\alpha_1 x_1 + ... + \alpha_n x_n = 0$, то
$\alpha_1 x_1 + ... \alpha_d x_d = -\alpha_{d+1} x_{d+1} - ... - \alpha_n x_n$ и
$\alpha_1 x_1 + ... \alpha_d x_d \in kerP$, a $-\alpha_{d+1} x_{d+1} - ... - \alpha_n x_n \in imP
\Rightarrow \alpha_1 x_1 + ... \alpha_d x_d = -\alpha_{d+1} x_{d+1} - ... - \alpha_n x_n = 0
\Rightarrow \alpha_1 = ... = \alpha_n  = 0$). То есть $x_1, ... , x_n$ - базис $V$ (так как
это система линейно независимых векторов из $V$, число векторов в которой равно размерности $V$) $\Rightarrow
\forall x \in V \exists !$ $u \in kerP, v \in im P: x = u + v$ (если $x = \alpha_1 x_1 + ... + \alpha_n x_n$, то
$u = \alpha_1 x_1 + ... + \alpha_d x_d, v = \alpha_{d+1} x_{d+1} + ... + \alpha_n x_n$)\\
\end {itemize}
Таким образом, $V = L \oplus M$ и $P$ - оператор проектирования на $L$ параллельно $M$, так как
$\forall x \in V: x = u + v; u \in kerP, v \in im P$ имеем $Px = Pu + Pv = Pv \in imP$. \\ $\blacksquare$ \\ \\ \\


% 11 задача
\subsection*{Задача 11:}
\addcontentsline{toc}{subsection}{Задача 11:}
\noindent Докажите, что для того чтобы матрицы одинаковых размеров были матрицами одного и того
же линейного оператора в каких-то парах базисов, необходимо и достаточно, чтобы они имели
одинаковый ранг. \\ \\
% Решение 11 задачи
\textbf{\underline{Решение:}} \\
1. $\Rightarrow:$ \\
Пусть $A: V \rightarrow W$ - линейный оператор и $[A]_{fe}, [A]_{hg}$ - его матрицы парах базисов
$e, g \in V; f, h \in W$ \\
Тогда $\forall x \in V: [A]_{fe} [x]_e = [Ax]_f = P_{fh} [Ax]_h = P_{fh} [A]_{hg} [x]_g = P_{fh} [A]_{hg} P_{ge} [x]_e$.
Значит $[A]_{fe} = P_{fh} [A]_{hg} P_{ge} \Rightarrow rank[A]_{fe} = rank[A]_{hg}$.
(здесь $P_{fh}$ - матрица перехода из базиса $h$ в базис $f$; $P_{ge}$ - матрица перехода из базиса $e$ в базис $g$) \\ \\
2. $\Leftarrow:$ \\
Пусть $B, C$ - матрицы одинаковых размеров ($A, B \in \mathbb{R}^{m \times n}$), такие что $rankB = rankC$. Из первого семестра знаем, что в таком случае
они эквивалентны, то есть $\exists$ невырожденные матрицы $P, Q: B = PCQ$. \\
Выберем произвольные линейные пространства $V, W$ - такие, что их размерности совпадают с размерами матриц ($dimV = n, dimW = m$),
а также произвольные базисы $e \in V, f \in W$. Построим линейный оператор $A: V \rightarrow W$ такой, что
$[A]_{fe} = B$ (просто определяем действие оператора на базисные векторы так, чтобы $Ae_i = b_{1i}f_1 + ... + b_{mi} f_m$) \\
Затем построим базисы $g \in V, h \in W$ так, чтобы матрицы $P, Q$ были матрицами перехода:
$Q = P_{ge}, P = P_{fh}$. Тогда $C = P^{-1}BQ^{-1} = P_{hf} [A]_{fe} P_{eg} = [A]_{hg} \Rightarrow$
матрицы $B, C$ являются матрицами одного оператора в парах базисов $e, f$ и $g, h$. \\ $\blacksquare$ \\ \\ \\


% 12 задача
\subsection*{Задача 12:}
\addcontentsline{toc}{subsection}{Задача 12:}
\noindent Докажите, что определитель и след квадратный матрицы являются инвариантами подобия. \\ \\
% Решение 12 задачи
\textbf{\underline{Решение:}} \\
Пусть $A, B \in \mathbb{R}^{n \times n}$ - подобные матрицы (то есть $A = PBP^{-1}$).
Тогда $|A| = |PBP^{-1}| = |P||B||P^{-1}| = |P||B|\frac{1}{|P|} = |B|$. \\
Также $trA = trP(BP^{-1}) = tr(BP^{-1})P = trB$. \\ $\blacksquare$ \\ \\ \\


% 13 задача
\subsection*{Задача 13:}
\addcontentsline{toc}{subsection}{Задача 13:}
\noindent Докажите, что характеристический многочлен квадратной матрицы является инвариантом подобия. \\ \\
% Решение 13 задачи
\textbf{\underline{Решение:}} \\
Пусть $A, B \in \mathbb{R}^{n \times n}$ - подобные матрицы (то есть $A = PBP^{-1}$).
Тогда $|A - \lambda I| = |PBP^{-1} - \lambda I| = |P(B - \lambda I)P^{-1}| = 
|P||B - \lambda I||P^{-1}| = |P||B - \lambda I|\frac{1}{|P|} = |B - \lambda I|$. \\ $\blacksquare$ \\ \\ \\


% 14 задача
\subsection*{Задача 14:}
\addcontentsline{toc}{subsection}{Задача 14:}
\noindent Найдите характеристический многочлен матрицы $A = $
$\begin{pmatrix}
    0 & 0 & \cdots & 0 & 1 \\
    0 & 0 & \cdots & 1 & 0 \\
    \vdots & \vdots & \cdots & \vdots & \vdots \\
    0 & 1 & \cdots & 0 & 0 \\
    1 & 0 & \cdots & 0 & 0
\end{pmatrix}_{n \times n}$ \\ \\
% Решение 14 задачи
\textbf{\underline{Решение:}} \\
1. $n = 2k, k \in \mathbb{N}$: \\
$|A - \lambda I| = 
\begin{vmatrix}
    -\lambda & 0 & \cdots & \cdots & 0 & 1 \\
    \vdots & \vdots & \cdots & \cdots & \vdots & \vdots \\ 
    0 & \cdots & -\lambda & 1 & \cdots & 0 \\
    0 & \cdots & 1 & -\lambda & \cdots & 0 \\
    \vdots & \vdots & \cdots & \cdots & \vdots & \vdots \\ 
    1 & 0 & \cdots & \cdots & 0 & -\lambda
\end{vmatrix} = 
\underbrace{\begin{vmatrix}
    -\lambda & 1 \\
    1 & -\lambda
\end{vmatrix}
\begin{vmatrix}
    -\lambda & 0 & \cdots & \cdots & 0 & 1 \\
    \vdots & \vdots & \cdots & \cdots & \vdots & \vdots \\ 
    0 & \cdots & -\lambda & 1 & \cdots & 0 \\
    0 & \cdots & 1 & -\lambda & \cdots & 0 \\
    \vdots & \vdots & \cdots & \cdots & \vdots & \vdots \\ 
    1 & 0 & \cdots & \cdots & 0 & -\lambda
\end{vmatrix}_{(n - 2) \times (n - 2)}}_{\text{применили теорему Лапласа по первой и последней строке}} = 
\underbrace{\begin{vmatrix}
    -\lambda & 1 \\
    1 & -\lambda
\end{vmatrix}^{\frac{n}{2}}}_{\text{продолжили Лапласом}} = (\lambda - 1)^{\frac{n}{2}}(\lambda + 1)^{\frac{n}{2}}$ \\ \\ \\
2. $n = 2k + 1, k \in \mathbb{N}$: \\
$|A - \lambda I| = 
\begin{vmatrix}
    -\lambda & 0 & \cdots & 0 & 1 \\
    \vdots & \vdots & \cdots & \vdots & \vdots \\ 
    0 & \cdots & 1 -\lambda & \cdots & 0 \\
    \vdots & \vdots & \cdots & \vdots & \vdots \\ 
    1 & 0 & \cdots & 0 & -\lambda
\end{vmatrix} =
\underbrace{(1 - \lambda)
\begin{vmatrix}
    -\lambda & 0 & \cdots & \cdots & 0 & 1 \\
    \vdots & \vdots & \cdots & \cdots & \vdots & \vdots \\ 
    0 & \cdots & -\lambda & 1 & \cdots & 0 \\
    0 & \cdots & 1 & -\lambda & \cdots & 0 \\
    \vdots & \vdots & \cdots & \cdots & \vdots & \vdots \\ 
    1 & 0 & \cdots & \cdots & 0 & -\lambda
\end{vmatrix}_{(n - 1) \times (n - 1)}}_{\text{Применили теорему Лапласа по центральной строке}} = 
\underbrace{(1 - \lambda)(\lambda - 1)^{\frac{n - 1}{2}}(\lambda + 1)^{\frac{n - 1}{2}}}_{\text{воспользовались первым пунктом решения}} = 
-(\lambda - 1)^{\frac{n + 1}{2}}(\lambda + 1)^{\frac{n - 1}{2}}$ \\ $\blacksquare$ \\ \\ \\


% 15 задача
\subsection*{Задача 15:}
\addcontentsline{toc}{subsection}{Задача 15:}
\noindent Найдите все инвариантные подпространства оператора дифференцирования в пространстве
всех вещественных многочленов. \\ \\
% Решение 15 задачи
\textbf{\underline{Решение:}} \\


% 16 задача
\subsection*{Задача 16:}
\addcontentsline{toc}{subsection}{Задача 16:}
\noindent Докажите, что число является собственным значением линейного оператора на конечномерном
пространстве в том и только том случае, когда оно является корнем его характеристического
многочлена. \\ \\
% Решение 16 задачи
\textbf{\underline{Решение:}} \\
1. $\Rightarrow:$ \\
Пусть $\lambda$ - собственное значение линейного оператора $A: V \rightarrow V$ над полем $\mathbb{P}$.
Тогда $\exists x \neq 0: Ax = \lambda x \Rightarrow (A - \lambda I)x = 0 \Rightarrow |A - \lambda I| = 0$. \\ \\
2. $\Leftarrow:$ \\
Пусть $\lambda$ - корень характеристического многочлена линейного оператора $A: V \rightarrow V$ над полем $\mathbb{P}$.
Тогда $|A - \lambda I| = 0 \Rightarrow \exists x \neq 0: (A - \lambda I)x = 0 \Rightarrow Ax = \lambda x
\Rightarrow \lambda$ - собственное значение $A$. \\ $\blacksquare$ \\ \\ \\


% 17 задача
\subsection*{Задача 17:}
\addcontentsline{toc}{subsection}{Задача 17:}
\noindent Линейный оператор действует в n-мерном пространстве над полем, содержащим все корни его
характеристического многочлена. Докажите, что при любом заранее предписанном порядке
корней существует базис пространства, в котором матрица оператора приобретает верхний
треугольный вид с главной диагональю, заполненной корнями в заранее предписанном порядке. \\ \\
% Решение 17 задачи
\textbf{\underline{Решение:}} \\


% 18 задача
\subsection*{Задача 18:}
\addcontentsline{toc}{subsection}{Задача 18:}
\noindent Докажите, что если матрицы $A$ и $B$ подобны, то для произвольного многочлена $f(\lambda)$ матрицы
$f(A)$ и $f(B)$ тоже подобны. \\ \\
% Решение 18 задачи
\textbf{\underline{Решение:}} \\
Пусть $A = PBP^{-1}$ и $f(\lambda) = a_0 + a_1 \lambda + ... + a_n \lambda^n$.
Тогда $f(A) = a_0I + a_1A + ... + a_nA^n = a_0I + a_1PBP^{-1} + ... + a_n(PBP^{-1})^n =
a_0PIP^{-1} + a_1PBP^{-1} + ... + a_nPB^nP^{-1} = P(a_0I + a_1B + ... + a_nB^n)P^{-1} = Pf(B)P^{-1}$ 
(воспользовались тем, что $(PBP^{-1})^n = \underbrace{PBP^{-1}PB \cdots PBP^{-1}}_{\text{n раз}} = PB^nP^{-1}$)\\ $\blacksquare$ \\ \\ \\


% 19 задача
\subsection*{Задача 19:}
\addcontentsline{toc}{subsection}{Задача 19:}
\noindent Докажите, что минимальный многочлен, аннулирующий квадратную матрицу, является 
делителем ее характеристического многочлена. \\ \\
% Решение 19 задачи
\textbf{\underline{Решение:}} \\
Пусть $A \in \mathbb{R}^{n \times n}$ - квадратная матрица, $f(\lambda)$ - ее минимальный многочлен, 
а $g(\lambda)$ - характеристический. \\
Предположим противное - пусть $g(\lambda) \mathrel{\vdots \hspace{-0.60em} \diagup} f(\lambda)$.
Тогда $g(\lambda) = f(\lambda)h(\lambda) + r(\lambda)$, $deg(r) < deg(f)$.\\Тогда
$g(A) = f(A)h(A) + r(A) \Rightarrow 0 = 0 \times h(A) + r(A) \Rightarrow r(A) = 0$\\
($g(A) = 0$ по теореме Кэли-Гамильтона, $f(A) = 0$ по условию) \\
То есть $r(A) = 0, deg(r) < deg(f) \Rightarrow f(\lambda)$ - не минимальный многочлен $\Rightarrow$ противоречие.
Значит $g(\lambda)$ $\vdots$ $f(\lambda)$. \\ $\blacksquare$ \\ \\ \\ 


% 20 задача
\subsection*{Задача 20:}
\addcontentsline{toc}{subsection}{Задача 20:}
\noindent Докажите, что любая квадратная матрица с элементами из произвольного поля аннулируется
своим характеристическим многочленом. \\ \\
% Решение 20 задачи
\textbf{\underline{Решение:}} \\


% 21 задача
\subsection*{Задача 21:}
\addcontentsline{toc}{subsection}{Задача 21:}
\noindent Докажите, что любой приведенный многочлен степени выше первой является характеристическим многочленом некоторой матрицы. \\ \\
% Решение 21 задачи
\textbf{\underline{Решение:}} \\


% 22 задача
\subsection*{Задача 22:}
\addcontentsline{toc}{subsection}{Задача 22:}
\noindent Преобразование $A \rightarrow PAP^{-1}$ будем называть элементарным преобразованием подобия, если
матрица $P$ является либо матрицей перестановки, либо матрицей вида $I + \gamma E_{kl}$ , где матрица
$E_{kl}$ отличается от нулевой матрицы только одним элементом — единицей в позиции $(k,l)$ при
$k \neq l$, а число $\gamma$ может быть произвольным и выбираетcя в зависимости от ситуации. Докажите,
что матрица порядка $n$ с помощью $O(n^2)$ элементарных преобразований подобия приводится к
верхнему почти треугольному виду. \\ \\
% Решение 22 задачи
\textbf{\underline{Решение:}} \\


% 23 задача
\subsection*{Задача 23:}
\addcontentsline{toc}{subsection}{Задача 23:}
\noindent Докажите, что алгебраическая кратность собственного значения не меньше его геометрической кратности. \\ \\
% Решение 23 задачи
\textbf{\underline{Решение:}} \\
Пусть $A: V \rightarrow V$ - линейный оператор над полем $\mathbb{P}$, $dimV = n$. \\
Пусть $\lambda_i$ - собственное значение $A$. Рассмотрим произвольный базис, в котором первые 
$l_i$ векторов принадлежат собственному пространству для $\lambda_i$ (т.е $ker(A - \lambda_i I)$).
В этом базисе матрица оператора имеет вид:
$
\begin{pmatrix}
    \lambda_i I_{l_i} & A_{12} \\
    0 & A_{22}
\end{pmatrix} \Rightarrow |A - \lambda I_n| = 
(\lambda_i - \lambda)^{l_i} |A_{22} - \lambda I_{n - l_i}| \Rightarrow$ алгебраическая кратность $\lambda_i$ не меньше $l_i$. \\ $\blacksquare$ \\ \\ \\


% 24 задача
\subsection*{Задача 24:}
\addcontentsline{toc}{subsection}{Задача 24:}
\noindent Докажите, что собственные векторы для попарно различных собственных значений линейно независимы. \\ \\
% Решение 24 задачи
\textbf{\underline{Решение:}} \\
Пусть $A: V \rightarrow V$ - линейный оператор над полем $\mathbb{P}$, $\lambda_1, ... , \lambda_m$ - попарно различные собственные значения $A$ 
и $x_1, ... , x_m$ - соответствующие собственные векторы. \\
Рассмотрим их линейную комбинацию $\alpha_1 x_1 + ... + \alpha_m x_m = 0$. Умножим это равенство на $\lambda_m$, 
а также подействуем на обе его стороны линейным оператором $A$. В результате получим два равенства: \\
$$
\begin{cases}
    \alpha_1 \lambda_1 x_1 + ... + \alpha_m \lambda_m x_m = 0 \\
    \alpha_1 \lambda_m x_1 + ... + \alpha_m \lambda_m x_m = 0
\end{cases}
$$
Вычитаем из первого второе
$$\alpha_1 (\lambda_1 - \lambda_m)x_1 + ... + \alpha_{m - 1} (\lambda_{m - 1} - \lambda_m)x_{m - 1} = 0$$
То есть если $x_1, ... , x_{m - 1}$ - линейно независимы, то $\alpha_1 = ... = \alpha_{m - 1} = 0 \Rightarrow
\alpha_m x_m = 0 \Rightarrow \alpha_m = 0$ (так как $x_m \neq 0$) \\
Далее применяем индукцию (база - 1 вектор очевидно линейно независим сам с собой). \\ $\blacksquare$ \\ \\ \\


% 25 задача
\subsection*{Задача 25:}
\addcontentsline{toc}{subsection}{Задача 25:}
\noindent Докажите, что если матрица порядка $n$ имеет $n$ попарно различных собственных значений, то она диагонализуема. \\ \\
% Решение 25 задачи
\textbf{\underline{Решение:}} \\
Пусть $A \in \mathbb{R}^{n \times n}$ - матрица с $n$ попарно различными собственными значениями $\lambda_1, ... , \lambda_n$ и 
$x_1, ... , x_n$ - соответствующие собственные векторы. \\
Из предыдущей задачи знаем, что векторы $x_1, ... , x_n$ - линейно независимы $\Rightarrow$ они образуют
базис $\mathbb{R}^n$. \\
Составим из них матрицу $X = (x_1, ... , x_n)$.
Тогда $AX = X
\begin{pmatrix}
    \lambda_1 & & \\
     & \ddots & \\
     & & \lambda_n
\end{pmatrix}
$ (так как $Ax_i = \lambda_i x_i$) $\Rightarrow XAX^{-1} =
\begin{pmatrix}
    \lambda_1 & & \\
     & \ddots & \\
     & & \lambda_n
\end{pmatrix}$ - диагональная матрица $\Rightarrow$ $A$ - диагонализуема. \\ $\blacksquare$ \\ \\ \\


\newpage
\section*{Расстояния, нормы, скалярные произведения, полиэдры.}
\addcontentsline{toc}{section}{Расстояния, нормы, скалярные произведения, полиэдры.}
\subsection*{Задача 1:}
\addcontentsline{toc}{subsection}{Задача 1:}
\noindent Докажите, что функция $p(x, y) = |x - y|/(1 + |x - y|)$ задает расстояние в вещественном пространстве $\mathbb{R}$. Будет ли пространство полным? \\ \\
\textbf{\underline{Решение:}} \\
Проверим свойства нормы:
\begin{enumerate}
    \item Симметричность - очевидна из симметричности модуля
    $$|x - y|/(1 + |x - y|) = |y - x|/(1 + |y - x|)$$
    \item Неотрицательность - очевидна из неотрицательности числителя и знаменателя $p(x, y) \ge 0$
    \item Неравенство треугольника -
    \begin{gather*}
        p(x, y) = \frac{|x - y|}{1 + |x - y|} = 1 - \frac{1}{1 + |x - y|} \le \text{(неравенство треугольника для модуля)} \le \\
        \le 1 - \frac{1}{1 + |x - z| + |z - y|} = \frac{|x - z| + |z - y|}{1 + |x - z| + |z - y|} = \\
        = \frac{|x - z|}{1 + |x - z| + |z - y|} + \frac{|z - y|}{1 + |x - z| + |z - y|} \le \frac{|x - z|}{1 + |x - z|} + \frac{|z - y|}{1 + |z - y|} = \\
        = p(x, z) + p(z, y)
    \end{gather*}
\end{enumerate}
$\blacksquare$ \\ \\ \\


\subsection*{Задача 2:}
\addcontentsline{toc}{subsection}{Задача 2:}
\noindent Докажите, что множество $S$ замкнуто в метрическом пространстве $M$ тогда и только тогда, когда дополнительное множество $M\backslash S$ открыто. \\ \\
\textbf{\underline{Решение:}} \\
Разобьем решение на два подпункта:
\begin{enumerate}
    \item Покажем достаточность. $S$ - замкнуто, что означает что оно содержит все свои предельные точки. Предположим, что $M\backslash S$ не является открытым, тогда $\exists x \in M\backslash S$, такая что в любой ее окрестности содержится хотя бы одна точка из $S$ (иначе $x$ - внутренняя) $\Rightarrow x$ - предельная для $S$, но $x \not\in S$ - противоречие с замкнутостью $S$. 
    \item Покажем необходимость. $M\backslash S$ - открыто, значит каждая точка этого мн-ва является внутренней. Предположим, что $S$ не замкнуто, тогда $\exists x \in M\backslash S : x$ - предельная для $S \Rightarrow$ в любой окрестности $x$ если точка из $S$ и $x$ не является внутренней - противоречие.
\end{enumerate}
$\blacksquare$ \\ \\ \\


\subsection*{Задача 3:}
\addcontentsline{toc}{subsection}{Задача 3:}
\noindent Пусть точками метрического пространства $M$ являются натуральные числа, а расстояние между $m$ и $n$ определяется как $\rho(m, n) = 1 + min(1/m, 1/n)$ при $m \not= n$ и $0$ при $m = n$. Докажите, что $M$ - полное метрическое пространство. Докажите также, что замкнутые шары $
\overline{M}(1, 1 + 1/2) \supset \overline{M}(1, 1 + 1/3) \supset \overline{M}(1, 1 + 1/4) \supset ...$ вложены, но имеют пустое пересечение. \\ \\
\textbf{\underline{Решение:}} \\
Разобьем решение на два подпункта:
\begin{enumerate}
    \item Покажем достаточность. $S$ - замкнуто, что означает что оно содержит все свои предельные точки. Предположим, что $M\backslash S$ не является открытым, тогда $\exists x \in M\backslash S$, такая что в любой ее окрестности содержится хотя бы одна точка из $S$ (иначе $x$ - внутренняя) $\Rightarrow x$ - предельная для $S$, но $x \not\in S$ - противоречие с замкнутостью $S$. 
    \item Покажем необходимость. $M\backslash S$ - открыто, значит каждая точка этого множ-ва является внутренней. Предположим, что $S$ не замкнуто, тогда $\exists x \in M\backslash S : x$ - предельная для $S \Rightarrow$ в любой окрестности $x$ если точка из $S$ и $x$ не является внутренней - противоречи.
\end{enumerate}
$\blacksquare$ \\ \\ \\



\end{document}
