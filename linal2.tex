\documentclass[a4paper,12pt,titlepage,final]{article}
% при подготовке финальной версии отчёта смените опцию draft на final

\usepackage[T1,T2A]{fontenc}     % форматы шрифтов
\usepackage[utf8x]{inputenc}     % кодировка символов, используемая в данном файле
\usepackage[russian]{babel}      % пакет русификации
\usepackage{tikz}                % для создания иллюстраций
\usepackage{pgfplots}            % для вывода графиков функций
\usepackage{geometry}		     % для настройки размера полей
\usepackage{indentfirst}         % для отступа в первом абзаце секции
\usepackage{listings}			 % для текста Make-файла
\usepackage{amsmath}			 % для формул
\usepackage{amssymb}			 % для символов

% выбираем размер листа А4, все поля ставим по 3см
\geometry{a4paper,left=30mm,top=30mm,bottom=30mm,right=30mm}
% добавляем точку после номера раздела
\renewcommand{\thesection}{\arabic{section}.}
\usepgfplotslibrary{fillbetween} % для изображения областей на графиках

% автор, дата и название документа
\title{Сборник задач и их решений по предмету "Линейная алгебра", 1 поток, 2 семестр}
\author{ Над этим дерьмом трудились\\Денисов Егор, Долгушев Глеб, Курбацкий Вячеслав из 201 группы}
\date{Когда-то в 2024 году...}
	
\begin{document}

\maketitle

% Оглавление
\newpage
\tableofcontents

% 1 раздел - линейные операторы и жорданки
\newpage
\section*{Линейные операторы и жорданки}
\addcontentsline{toc}{section}{Линейные операторы и жорданки}

% 1 задача
\subsection*{Задача 1:}
\addcontentsline{toc}{subsection}{Задача 1:}
\noindent Докажите, что линейная операция линейных операторов является линейным оператором. \\ \\
% Решение 1 задачи
\textbf{\underline{Решение:}} \\
Пусть $A, B: V \rightarrow W$ - линейные операторы над полем $\mathbb{P}$, a $C = \alpha A + \beta B$ - их линейная комбинация.\\
Тогда $\forall x, y \in V$ и $\forall \delta, \gamma \in \mathbb{P}$:
$C(\delta x + \gamma y) = (\alpha A + \beta B) (\delta x + \gamma y) = \alpha A(\delta x + \gamma y) + \beta B(\delta x + \gamma y) =
\alpha (\delta Ax + \gamma Ay) + \beta (\delta Bx + \gamma By) = \delta (\alpha Ax + \beta Bx) + \gamma (\alpha Ay + \beta By) =
\delta (\alpha A + \beta B)x + \gamma (\alpha A + \beta B)y = \delta Cx + \gamma Cy$ \\ $\blacksquare$ \\ \\ \\

% 2 задача
\subsection*{Задача 2:}
\addcontentsline{toc}{subsection}{Задача 2:}
\noindent Докажите, что произведение линейных операторов является линейным оператором. \\ \\
% Решение 2 задачи
\textbf{\underline{Решение:}} \\
Пусть $A: V \rightarrow W$ и $B: W \rightarrow U$ - линейные операторы над полем $\mathbb{P}$, a $C = B \circ A$ - их произведение.\\
Тогда $\forall x, y \in V$ и $\forall \alpha, \beta \in \mathbb{P}$:
$C(\alpha x + \beta y) = B(A(\alpha x + \beta y)) = B(\alpha Ax + \beta Ay) = \alpha B(Ax) + \beta B(Ay) = \alpha Cx + \beta Cy$ \\ $\blacksquare$ \\ \\ \\

% 3 задача
\subsection*{Задача 3:}
\addcontentsline{toc}{subsection}{Задача 3:}
\noindent Докажите, что если линейный оператор обратим, то обратный оператор определен однозначно и является линейным. \\ \\
% Решение 3 задачи
\textbf{\underline{Решение:}} \\
1. Единственность: \\
Пусть $A: V \rightarrow W$ - обратимый линейный оператор над полем $\mathbb{P}$, a $B$ и $C$ - его обратные операторы.\\
Тогда $\forall x \in W: Bx = B(AC)x = (BA)Cx = Cx \Rightarrow B = C$ \\ \\
2. Линейность: \\
Пусть $A: V \rightarrow W$ - обратимый линейный оператор над полем $\mathbb{P}$, a $B$ - его обратный оператор.\\
$\forall y_1, y_2 \in W$ имеют вид $ y_1 = Ax_1, y_2 = Ax_2, x_1, x_2 \in V \Rightarrow x_1 = By_1, x_2 = By_2$ \\
Тогда $A(\alpha x_1 + \beta x_2) = \alpha (AB)y_1 + \beta (AB)y_2 = \alpha y_1 + \beta y_2 \Rightarrow B(\alpha y_1 + \beta y_2) =$
(так как $B$ - обратный, то из $Ax = y$ следует $By = x$) $= \alpha x_1 + \beta x_2 = \alpha By_1 + \beta By_2$ \\ $\blacksquare$ \\ \\ \\


% 4 задача
\subsection*{Задача 4:}
\addcontentsline{toc}{subsection}{Задача 4:}
\noindent Пусть $V$ и $W$ - конечномерные пространства над общим полем. Докажите, что для обратимости линейного оператора $A: V \rightarrow W$
необходимо и достаточно выполнение условий $dim V = dim W$ и $ker A = \{0\}$ \\ \\
% Решение 4 задачи
\textbf{\underline{Решение:}} \\
1. $\Rightarrow:$ \\
Пусть $A: V \rightarrow W$ - обратимый линейный оператор над полем $\mathbb{P}$ и $B$ - его обратный оператор. \\
Если $kerA \neq \{0\}$, то $\exists x \neq 0: Ax = 0 \Rightarrow x = B(Ax) = B0 = 0 \Rightarrow x = 0$ - противоречие. \\
Так как $AV = W$, то $dimW \leq dimV$. Поясним этот момент - пусть $x_1, x_2, ..., x_n$ - базис $V$.
Тогда $\forall y \in W: y = Ax = A(\alpha_1 x_1 + ... + \alpha_n x_n) = \alpha_1 (Ax_1) + ... + \alpha_n  (Ax_n)$.
То есть все векторы из $W$ выражаются через векторы $Ax_1, ... , Ax_n$. Значит в базисе $W$ векторов не больше, чем $n$.
Аналогично $BW = V \Rightarrow dimV \leq dimW$. Значит $dimV = dimW$. \\ \\
2. $\Leftarrow$: \\
Пусть $dimV = dimW = n$ и $kerA = \{0\}$. Тогда по теореме о ранге и дефекте - $rankA = n$.
Так как $imA \subseteq W$ и $dim(imA) = dimW$, то $imA = W \Rightarrow$ $A$ - сюръективный. \\
Так как $kerA = \{0\}$, то $A$ - инъективный (иначе, если $\exists x_1 \neq x_2: Ax_1 = Ax_2$, то $A(x_1 - x_2) = Ax_3 = 0$, где $x_3 \neq 0 \Rightarrow
kerA \neq \{0\}$ - противоречие). \\
То есть $\forall y \in W$ $\exists ! x \in V: Ax = y \Rightarrow$ построим оператор $B: W \rightarrow V$ таким образом: $By = x$, где $x$ - такой, что $Ax = y$.
Построенный оператор $B$ по определению - обратный для $A$ (так как $(AB)y = y$ и $(BA)x = x \forall x \in V, y \in W$). \\ $\blacksquare$ \\ \\ \\


% 5 задача
\subsection*{Задача 5:}
\addcontentsline{toc}{subsection}{Задача 5:}
\noindent Докажите, что ядро и образ линейного оператора являются его инвариантными подпространствами. \\ \\
% Решение 5 задачи
\textbf{\underline{Решение:}} \\
Пусть $A: V \rightarrow V$ - линейный оператор над полем $\mathbb{P}$. \\
$\forall x \in kerA: Ax = 0 \in kerA \Rightarrow kerA$ - инвариантное подпространство. \\
$\forall y \in imA: Ay \in imA$ по определению образа $\Rightarrow imA$ - инвариантное подпространство. \\ $\blacksquare$ \\ \\ \\


% 6 задача
\subsection*{Задача 6:}
\addcontentsline{toc}{subsection}{Задача 6:}
\noindent Докажите, что сумма ранга и дефекта линейного оператора равна размерности его области определения. \\ \\
% Решение 6 задачи
\textbf{\underline{Решение:}} \\
Пусть $A: V \rightarrow W$ - линейный оператор над полем $\mathbb{P}$. \\
Пусть $defA = d$ и $v_1, ... , v_d$ - базис $kerA$. Дополним его до базиса всего пространства $V$ векторами $v_{d+1}, ... , v_n$. \\
Тогда $\forall x \in V: x = \alpha_1 v_1 + ... + \alpha_d v_d + \alpha_{d+1} v_{d+1} + ... + \alpha_n v_n \Rightarrow
Ax = A(\alpha_1 v_1 + ... + \alpha_d v_d + \alpha_{d+1} v_{d+1} + ... + \alpha_n v_n) =
\alpha_1 Av_1 + ... + \alpha_d Av_d + \alpha_{d+1} Av_{d+1} + ... + \alpha_n Av_n = \alpha_{d+1} Av_{d+1} + ... + \alpha_n Av_n$ (так как $Av_1 = ... = Av_d = 0$) \\
Таким образом, $imA = L(Av_{d+1}, ... , Av_n)$. Докажем, что эти векторы линейно независимы: \\
Пусть $\alpha_{d+1} Av_{d+1} + ... + \alpha_n Av_n = 0 \Rightarrow A(\alpha_{d+1} v_{d+1} + ... + \alpha_n v_n) = 0
\Rightarrow \alpha_{d+1} v_{d+1} + ... + \alpha_n v_n \in kerA \Rightarrow \alpha_{d+1} v_{d+1} + ... + \alpha_n v_n = \alpha_1 v_1 + ... + \alpha_d v_d \Rightarrow
\alpha_1 v_1 + ... + \alpha_d v_d - \alpha_{d+1} v_{d+1} - ... - \alpha_n v_n = 0 \Rightarrow \alpha_{d+1} = ... = \alpha_n = 0$ (так как $v_1, ... , v_n$ - линейно независимы).
То есть из $\alpha_{d+1} Av_{d+1} + ... + \alpha_n Av_n = 0$ следует $\alpha_{d+1} = ... = \alpha_n = 0 \Rightarrow Av_{d+1}, ... , Av_n$ - линейно независимые
$\Rightarrow rankA = n - d \Rightarrow rankA + defA = dimV$. \\ $\blacksquare$ \\ \\ \\


% 7 задача
\subsection*{Задача 7:}
\addcontentsline{toc}{subsection}{Задача 7:}
\noindent Докажите, что если сумма ядер двух линейных операторов, действующих на одном пространстве, совпадает с этим пространством,
то образ суммы этих операторов равен сумме их образов. \\ \\
% Решение 7 задачи
\textbf{\underline{Решение:}} \\
Пусть $A, B: V \rightarrow V$ - линейные операторы над полем $\mathbb{P}$ и $kerA + kerB = V$. \\
Пусть $x \in im(A + B) \Rightarrow \exists y \in V: (A + B)y = x \Rightarrow Ay + By = x \Rightarrow x \in imA + imB
\Rightarrow im(A + B) \subseteq imA + imB$. \\
Пусть $x \in imA + imB \Rightarrow \exists y_1, y_2 \in V: x = Ay_1 + By_2$. Так как $kerA + kerB = V$, то
$y_1 = u_1 + v_1, y_2 = u_2 + v_2$, где $u_1, u_2 \in kerA; v_1, v_2 \in kerB$. Значит $x = A(u_1 + v_1) + B(u_2 + v_2) =
Av_1 + Bu_2 = Av_1 + 0 + Bu_2 + 0 = Av_1 + Au_2 + Bu_2 + Bv_1 = (A + B)(v_1 + u_2) \Rightarrow x \in im(A + B)
\Rightarrow imA + imB \subseteq im(A + B)$.\\
То есть $im(A + B) = imA + imB$. \\ $\blacksquare$ \\ \\ \\


% 8 задача
\subsection*{Задача 8:}
\addcontentsline{toc}{subsection}{Задача 8:}
\noindent Линейный оператор $A: V \rightarrow V$ удовлетворяет равенству $A^m = 0$. Докажите, что оператор
$I - A$ обратим. \\ \\
% Решение 8 задачи
\textbf{\underline{Решение:}} \\
Пусть $A: V \rightarrow V$ - линейный оператор над полем $\mathbb{P}$ и $A^m = 0$. \\
Тогда $(I - A)(I + A + A^2 + ... + A^{m-1}) = (I + A + A^2 + ... + A^{m-1})(I - A) = I - A^m = I \Rightarrow I - A$ - обратим. \\ $\blacksquare$ \\ \\ \\


% 9 задача
\subsection*{Задача 9:}
\addcontentsline{toc}{subsection}{Задача 9:}
\noindent Линейные операторы $A$ и $B$ таковы, что оператор $A + B$ - обратимый. Докажите, что операторы
$P = (A + B)^{-1}A$ и $Q = (A + B)^{-1}B$ коммутируют. \\ \\
% Решение 9 задачи
\textbf{\underline{Решение:}} \\
Заметим, что $P + Q = (A + B)^{-1}(A + B) = I \Rightarrow P = I - Q$. \\
Тогда $PQ = (I - Q)Q = Q - Q^2 = Q(I - Q) = QP$. \\ $\blacksquare$ \\ \\ \\ 


% 10 задача
\subsection*{Задача 10:}
\addcontentsline{toc}{subsection}{Задача 10:}
\noindent Докажите, что для того, чтобы линейный оператор $P: V \rightarrow V$ был оператором проектирования,
необходимо и достаточно, чтобы $P^2 = P$. \\ \\
% Решение 10 задачи
\textbf{\underline{Решение:}} \\
1. $\Rightarrow:$ \\
Пусть $V = L \oplus M$ и $P$ - оператор проектирования на $L$ параллельно $M$, то есть
$\forall x \in V: x = u + v; u \in L, v \in M$ имеем $Px = u$. \\
Тогда $\forall x \in V: x = u + v$ выполняется $P^2x = P(Px) = P(u) = u = Px \Rightarrow P^2 = P$. \\ \\
2. $\Leftarrow:$ \\
Пусть $P^2 = P$. Тогда $V = L \oplus M, где L = imP, M = kerP$. Докажем это: \\
\begin {itemize}
\item По теореме о ранге и дефекте знаем, что $dim(imP) + dim(kerP) = rankP + defP = dim V$ \\
\item Если $x \in kerP \cap imP$, то $Px = 0$ и $x = Py \Rightarrow Px = P^2y = Py = x = 0 \Rightarrow kerP \cap imP = \{0\}$\\
\item Тогда, если $x_1, ... , x_d$ - базис $kerP$, а $x_{d+1}, ... , x_n$ - базис $imP$, то
$x_1, ... , x_n$ все вместе линейно независимы (так как если $\alpha_1 x_1 + ... + \alpha_n x_n = 0$, то
$\alpha_1 x_1 + ... \alpha_d x_d = -\alpha_{d+1} x_{d+1} - ... - \alpha_n x_n$ и
$\alpha_1 x_1 + ... \alpha_d x_d \in kerP$, a $-\alpha_{d+1} x_{d+1} - ... - \alpha_n x_n \in imP
\Rightarrow \alpha_1 x_1 + ... \alpha_d x_d = -\alpha_{d+1} x_{d+1} - ... - \alpha_n x_n = 0
\Rightarrow \alpha_1 = ... = \alpha_n  = 0$). То есть $x_1, ... , x_n$ - базис $V$ (так как
это система линейно независимых векторов из $V$, число векторов в которой равно размерности $V$) $\Rightarrow
\forall x \in V \exists !$ $u \in kerP, v \in im P: x = u + v$ (если $x = \alpha_1 x_1 + ... + \alpha_n x_n$, то
$u = \alpha_1 x_1 + ... + \alpha_d x_d, v = \alpha_{d+1} x_{d+1} + ... + \alpha_n x_n$)\\
\end {itemize}
Таким образом, $V = L \oplus M$ и $P$ - оператор проектирования на $L$ параллельно $M$, так как
$\forall x \in V: x = u + v; u \in kerP, v \in im P$ имеем $Px = Pu + Pv = Pv \in imP$. \\ $\blacksquare$ \\ \\ \\


% 11 задача
\subsection*{Задача 11:}
\addcontentsline{toc}{subsection}{Задача 11:}
\noindent Докажите, что для того чтобы матрицы одинаковых размеров были матрицами одного и того
же линейного оператора в каких-то парах базисов, необходимо и достаточно, чтобы они имели
одинаковый ранг. \\ \\
% Решение 11 задачи
\textbf{\underline{Решение:}} \\
1. $\Rightarrow:$ \\
Пусть $A: V \rightarrow W$ - линейный оператор и $[A]_{fe}, [A]_{hg}$ - его матрицы парах базисов
$e, g \in V; f, h \in W$ \\
Тогда $\forall x \in V: [A]_{fe} [x]_e = [Ax]_f = P_{fh} [Ax]_h = P_{fh} [A]_{hg} [x]_g = P_{fh} [A]_{hg} P_{ge} [x]_e$.
Значит $[A]_{fe} = P_{fh} [A]_{hg} P_{ge} \Rightarrow rank[A]_{fe} = rank[A]_{hg}$.
(здесь $P_{fh}$ - матрица перехода из базиса $h$ в базис $f$; $P_{ge}$ - матрица перехода из базиса $e$ в базис $g$) \\ \\
2. $\Leftarrow:$ \\
Пусть $B, C$ - матрицы одинаковых размеров ($A, B \in \mathbb{R}^{m \times n}$), такие что $rankB = rankC$. Из первого семестра знаем, что в таком случае
они эквивалентны, то есть $\exists$ невырожденные матрицы $P, Q: B = PCQ$. \\
Выберем произвольные линейные пространства $V, W$ - такие, что их размерности совпадают с размерами матриц ($dimV = n, dimW = m$),
а также произвольные базисы $e \in V, f \in W$. Построим линейный оператор $A: V \rightarrow W$ такой, что
$[A]_{fe} = B$ (просто определяем действие оператора на базисные векторы так, чтобы $Ae_i = b_{1i}f_1 + ... + b_{mi} f_m$) \\
Затем построим базисы $g \in V, h \in W$ так, чтобы матрицы $P, Q$ были матрицами перехода:
$Q = P_{ge}, P = P_{fh}$. Тогда $C = P^{-1}BQ^{-1} = P_{hf} [A]_{fe} P_{eg} = [A]_{hg} \Rightarrow$
матрицы $B, C$ являются матрицами одного оператора в парах базисов $e, f$ и $g, h$. \\ $\blacksquare$ \\ \\ \\


% 12 задача
\subsection*{Задача 12:}
\addcontentsline{toc}{subsection}{Задача 12:}
\noindent Докажите, что определитель и след квадратный матрицы являются инвариантами подобия. \\ \\
% Решение 12 задачи
\textbf{\underline{Решение:}} \\
Пусть $A, B \in \mathbb{R}^{n \times n}$ - подобные матрицы (то есть $A = PBP^{-1}$).
Тогда $|A| = |PBP^{-1}| = |P||B||P^{-1}| = |P||B|\frac{1}{|P|} = |B|$. \\
Также $trA = trP(BP^{-1}) = tr(BP^{-1})P = trB$. \\ $\blacksquare$ \\ \\ \\


% 13 задача
\subsection*{Задача 13:}
\addcontentsline{toc}{subsection}{Задача 13:}
\noindent Докажите, что характеристический многочлен квадратной матрицы является инвариантом подобия. \\ \\
% Решение 13 задачи
\textbf{\underline{Решение:}} \\
Пусть $A, B \in \mathbb{R}^{n \times n}$ - подобные матрицы (то есть $A = PBP^{-1}$).
Тогда $|A - \lambda I| = |PBP^{-1} - \lambda I| = |P(B - \lambda I)P^{-1}| = 
|P||B - \lambda I||P^{-1}| = |P||B - \lambda I|\frac{1}{|P|} = |B - \lambda I|$. \\ $\blacksquare$ \\ \\ \\


% 14 задача
\subsection*{Задача 14:}
\addcontentsline{toc}{subsection}{Задача 14:}
\noindent Найдите характеристический многочлен матрицы $A = $
$\begin{pmatrix}
    0 & 0 & \cdots & 0 & 1 \\
    0 & 0 & \cdots & 1 & 0 \\
    \vdots & \vdots & \cdots & \vdots & \vdots \\
    0 & 1 & \cdots & 0 & 0 \\
    1 & 0 & \cdots & 0 & 0
\end{pmatrix}_{n \times n}$ \\ \\
% Решение 14 задачи
\textbf{\underline{Решение:}} \\
1. $n = 2k, k \in \mathbb{N}$: \\
$|A - \lambda I| = 
\begin{vmatrix}
    -\lambda & 0 & \cdots & \cdots & 0 & 1 \\
    \vdots & \vdots & \cdots & \cdots & \vdots & \vdots \\ 
    0 & \cdots & -\lambda & 1 & \cdots & 0 \\
    0 & \cdots & 1 & -\lambda & \cdots & 0 \\
    \vdots & \vdots & \cdots & \cdots & \vdots & \vdots \\ 
    1 & 0 & \cdots & \cdots & 0 & -\lambda
\end{vmatrix} = 
\underbrace{\begin{vmatrix}
    -\lambda & 1 \\
    1 & -\lambda
\end{vmatrix}
\begin{vmatrix}
    -\lambda & 0 & \cdots & \cdots & 0 & 1 \\
    \vdots & \vdots & \cdots & \cdots & \vdots & \vdots \\ 
    0 & \cdots & -\lambda & 1 & \cdots & 0 \\
    0 & \cdots & 1 & -\lambda & \cdots & 0 \\
    \vdots & \vdots & \cdots & \cdots & \vdots & \vdots \\ 
    1 & 0 & \cdots & \cdots & 0 & -\lambda
\end{vmatrix}_{(n - 2) \times (n - 2)}}_{\text{применили теорему Лапласа по первой и последней строке}} = 
\underbrace{\begin{vmatrix}
    -\lambda & 1 \\
    1 & -\lambda
\end{vmatrix}^{\frac{n}{2}}}_{\text{продолжили Лапласом}} = (\lambda - 1)^{\frac{n}{2}}(\lambda + 1)^{\frac{n}{2}}$ \\ \\ \\
2. $n = 2k + 1, k \in \mathbb{N}$: \\
$|A - \lambda I| = 
\begin{vmatrix}
    -\lambda & 0 & \cdots & 0 & 1 \\
    \vdots & \vdots & \cdots & \vdots & \vdots \\ 
    0 & \cdots & 1 -\lambda & \cdots & 0 \\
    \vdots & \vdots & \cdots & \vdots & \vdots \\ 
    1 & 0 & \cdots & 0 & -\lambda
\end{vmatrix} =
\underbrace{(1 - \lambda)
\begin{vmatrix}
    -\lambda & 0 & \cdots & \cdots & 0 & 1 \\
    \vdots & \vdots & \cdots & \cdots & \vdots & \vdots \\ 
    0 & \cdots & -\lambda & 1 & \cdots & 0 \\
    0 & \cdots & 1 & -\lambda & \cdots & 0 \\
    \vdots & \vdots & \cdots & \cdots & \vdots & \vdots \\ 
    1 & 0 & \cdots & \cdots & 0 & -\lambda
\end{vmatrix}_{(n - 1) \times (n - 1)}}_{\text{Применили теорему Лапласа по центральной строке}} = 
\underbrace{(1 - \lambda)(\lambda - 1)^{\frac{n - 1}{2}}(\lambda + 1)^{\frac{n - 1}{2}}}_{\text{воспользовались первым пунктом решения}} = 
-(\lambda - 1)^{\frac{n + 1}{2}}(\lambda + 1)^{\frac{n - 1}{2}}$ \\ $\blacksquare$ \\ \\ \\


% 15 задача
\subsection*{Задача 15:}
\addcontentsline{toc}{subsection}{Задача 15:}
\noindent Найдите все инвариантные подпространства оператора дифференцирования в пространстве
всех вещественных многочленов. \\ \\
% Решение 15 задачи
\textbf{\underline{Решение:}} \\
Заметим, что при действии оператора дифференцирования степень многочлена понижается на 1. Значит, все пространства 
многочленов степени не выше $n$ будут инвариантными $\forall n \in \mathbb{N} \cup \{0\}$. \\
Докажем, что других инвариантных пространств нет:\\
Пусть $M$ - произвольное инвариантное подпространство оператора дифференцирования и $f$ - многочлен максимальной 
для $M$ степени $m$. Дифференцируя его, получим многочлены степеней $0, 1, ... , m - 1$, которые 
лежат в $M$ в силу того, что это инвариантное подпространство. Таким образом, $M$ совпадает с пространством многочленов 
степени не выше $m$. \\ $\blacksquare$ \\ \\ \\


% 16 задача
\subsection*{Задача 16:}
\addcontentsline{toc}{subsection}{Задача 16:}
\noindent Докажите, что число является собственным значением линейного оператора на конечномерном
пространстве в том и только том случае, когда оно является корнем его характеристического
многочлена. \\ \\
% Решение 16 задачи
\textbf{\underline{Решение:}} \\
1. $\Rightarrow:$ \\
Пусть $\lambda$ - собственное значение линейного оператора $A: V \rightarrow V$ над полем $\mathbb{P}$.
Тогда $\exists x \neq 0: Ax = \lambda x \Rightarrow (A - \lambda I)x = 0 \Rightarrow |A - \lambda I| = 0$. \\ \\
2. $\Leftarrow:$ \\
Пусть $\lambda$ - корень характеристического многочлена линейного оператора $A: V \rightarrow V$ над полем $\mathbb{P}$.
Тогда $|A - \lambda I| = 0 \Rightarrow \exists x \neq 0: (A - \lambda I)x = 0 \Rightarrow Ax = \lambda x
\Rightarrow \lambda$ - собственное значение $A$. \\ $\blacksquare$ \\ \\ \\


% 17 задача
\subsection*{Задача 17:}
\addcontentsline{toc}{subsection}{Задача 17:}
\noindent Линейный оператор действует в n-мерном пространстве над полем, содержащим все корни его
характеристического многочлена. Докажите, что при любом заранее предписанном порядке
корней существует базис пространства, в котором матрица оператора приобретает верхний
треугольный вид с главной диагональю, заполненной корнями в заранее предписанном порядке. \\ \\
% Решение 17 задачи
\textbf{\underline{Решение:}} \\
Пусть $A: V \rightarrow V$ - линейный оператор, действующий над полем $\mathbb{P}$, $dimV = n$, 
$\lambda_1, ... , \lambda_n \in \mathbb{P}$ - корни характеристического многочлена $A$ с учетом кратностей. \\
Пусть вектор $e_1$ - собственный для собственного значения $\lambda_1$. Достроим его до базиса 
$e_1, ... , e_n$. Тогда в этом базисе матрица оператора имеет вид: \\
$
\begin{pmatrix}
    \lambda_1 & \times & \cdots & \times \\
    0 & \times & & \vdots \\
    \vdots & & \ddots & \vdots \\
    0 & \cdots & \cdots & \times
\end{pmatrix} = 
\begin{pmatrix}
    \lambda_1 & \times \\
    0 & B
\end{pmatrix}, B \in \mathbb{R}^{(n - 1) \times (n - 1)}$ \\


% 18 задача
\subsection*{Задача 18:}
\addcontentsline{toc}{subsection}{Задача 18:}
\noindent Докажите, что если матрицы $A$ и $B$ подобны, то для произвольного многочлена $f(\lambda)$ матрицы
$f(A)$ и $f(B)$ тоже подобны. \\ \\
% Решение 18 задачи
\textbf{\underline{Решение:}} \\
Пусть $A = PBP^{-1}$ и $f(\lambda) = a_0 + a_1 \lambda + ... + a_n \lambda^n$.
Тогда $f(A) = a_0I + a_1A + ... + a_nA^n = a_0I + a_1PBP^{-1} + ... + a_n(PBP^{-1})^n =
a_0PIP^{-1} + a_1PBP^{-1} + ... + a_nPB^nP^{-1} = P(a_0I + a_1B + ... + a_nB^n)P^{-1} = Pf(B)P^{-1}$ 
(воспользовались тем, что $(PBP^{-1})^n = \underbrace{PBP^{-1}PB \cdots PBP^{-1}}_{\text{n раз}} = PB^nP^{-1}$)\\ $\blacksquare$ \\ \\ \\


% 19 задача
\subsection*{Задача 19:}
\addcontentsline{toc}{subsection}{Задача 19:}
\noindent Докажите, что минимальный многочлен, аннулирующий квадратную матрицу, является 
делителем ее характеристического многочлена. \\ \\
% Решение 19 задачи
\textbf{\underline{Решение:}} \\
Пусть $A \in \mathbb{R}^{n \times n}$ - квадратная матрица, $f(\lambda)$ - ее минимальный многочлен, 
а $g(\lambda)$ - характеристический. \\
Предположим противное - пусть $g(\lambda) \mathrel{\vdots \hspace{-0.60em} \diagup} f(\lambda)$.
Тогда $g(\lambda) = f(\lambda)h(\lambda) + r(\lambda)$, $deg(r) < deg(f)$.\\Тогда
$g(A) = f(A)h(A) + r(A) \Rightarrow 0 = 0 \times h(A) + r(A) \Rightarrow r(A) = 0$\\
($g(A) = 0$ по теореме Кэли-Гамильтона, $f(A) = 0$ по условию) \\
То есть $r(A) = 0, deg(r) < deg(f) \Rightarrow f(\lambda)$ - не минимальный многочлен $\Rightarrow$ противоречие.
Значит $g(\lambda)$ $\vdots$ $f(\lambda)$. \\ $\blacksquare$ \\ \\ \\ 


% 20 задача
\subsection*{Задача 20:}
\addcontentsline{toc}{subsection}{Задача 20:}
\noindent Докажите, что любая квадратная матрица с элементами из произвольного поля аннулируется
своим характеристическим многочленом. \\ \\
% Решение 20 задачи
\textbf{\underline{Решение:}} \\
Пусть $A \in \mathbb{P}^{n \times n}$ - квадратная матрица порядка $n$ с элементами из поля $\mathbb{P}$ и 
$\lambda_1, ... , \lambda_n$ - собственные значения $A$. \\ \\
Знаем, что $A = S^{-1}TS$, где 
$T = 
\begin{pmatrix}
    \lambda_1 & \cdots & \cdots \\
    \vdots & \ddots & \vdots \\
    0 & \cdots & \lambda_n
\end{pmatrix}
 \Rightarrow A^n = S^{-1}T^nS \Rightarrow \forall f(x): f(A) = S^{-1}f(T)S$ \\ \\
 Пусть $f(\lambda)$ - характеристический многочлен матрицы $A$. Тогда $$f(\lambda) = 
 (\lambda_1 - \lambda)...(\lambda_n - \lambda) \Rightarrow f(T) = 
 (\lambda_1 I - T)...(\lambda_n I - T)$$\\
 Обозначим $M_i = \lambda_i I - T$ и заметим, что 
 $M_i = 
 \begin{pmatrix}
    \times & \cdots & \cdots & \cdots & \times \\
     & \ddots & \cdots & \cdots & \vdots \\
     &  & 0 & \cdots & \times \\
     &  &  & \ddots & \vdots \\
     &  &  &  & \times
 \end{pmatrix}
 $ - верхнетреугольная матрица с нулем в позиции $(i, i)$. \\
Обычным умножением матриц проверяется, что $f(T) = M_1 \times M_2 \times ... \times M_n = 0$, так как 
$M_1 \times M_2 = 
\begin{pmatrix}
    0 & \cdots & \cdots & \times \\
     & \ddots & \cdots & \vdots \\
     &  & \ddots & \vdots \\
     &  &  & \times
\end{pmatrix} \times
\begin{pmatrix}
    \times & \cdots & \cdots & \times \\
     & 0 & \cdots & \times \\
     &  & \ddots & \vdots \\
     &  &  & \times
 \end{pmatrix} = 
 \begin{pmatrix}
    0 & 0 & \cdots & \times \\
     & 0 & \cdots & \times \\
     &  & \ddots & \vdots \\
     &  &  & \times
 \end{pmatrix}
 $, $M_1 \times ... \times M_k$ - аналогично будет верхнетреугольной матрицей с нулевой подматрицей $k \times k \Rightarrow
f(A) = S^{-1}0S = 0$ \\     
Если какие-то из собственных значений не принадлежат $\mathbb{P}$, то расширим поле $\mathbb{P}$
до поля разложения характеристического многочлена матрицы $A$ и рассмотрим ее как матрицу над этим, более широким, полем. \\
$\blacksquare$ \\ \\ \\



% 21 задача
\subsection*{Задача 21:}
\addcontentsline{toc}{subsection}{Задача 21:}
\noindent Докажите, что любой приведенный многочлен степени выше первой является характеристическим многочленом некоторой матрицы. \\ \\
% Решение 21 задачи
\textbf{\underline{Решение:}} \\
Пусть наш приведенный многочлен имеет вид $f(\lambda) = \lambda_n + a_{n - 1} \lambda^{n - 1} + ... + a_1 \lambda + a_0 \in 
\mathbb{P}[x]$ \\
Рассмотрим матрицу 
$A_f = 
\begin{pmatrix}
    0 & 0 & 0 & \cdots & 0 & -a_0 \\
    1 & 0 & 0 & \cdots & 0 & -a_1 \\
    0 & 1 & 0 & \cdots & 0 & -a_2 \\  
    \cdots & \cdots & \cdots & \cdots & \cdots & \cdots \\
    0 & 0 & 0 & \cdots & 0 & -a_{n - 2} \\  
    0 & 0 & 0 & \cdots & 1 & -a_{n - 1}
\end{pmatrix}
\in \mathbb{P}^{n \times n}$ - матрица Фробениуса для многочлена $f$. \\ \\
Посчитаем характеристический многочлен $A_f$: \\
$|A_f - \lambda I| = 
\begin{vmatrix}
    -\lambda & 0 & 0 & \cdots & 0 & -a_0 \\
    1 & -\lambda & 0 & \cdots & 0 & -a_1 \\
    0 & 1 & -\lambda & \cdots & 0 & -a_2 \\  
    \cdots & \cdots & \cdots & \cdots & \cdots & \cdots \\
    0 & 0 & 0 & \cdots & -\lambda & -a_{n - 2} \\  
    0 & 0 & 0 & \cdots & 1 & -a_{n - 1} - \lambda
\end{vmatrix}
 =$ (прибавим к 1 строке $i$ строку, умноженную на $\lambda^{i - 1}$  $\forall i = 2, ... , n$) $= 
 \begin{vmatrix}
    0 & 0 & 0 & \cdots & 0 & -a_0 - a_1 \lambda - ... - a_{n - 1} \lambda^{n - 1} - \lambda^n \\
    1 & -\lambda & 0 & \cdots & 0 & -a_1 \\
    0 & 1 & -\lambda & \cdots & 0 & -a_2 \\  
    \cdots & \cdots & \cdots & \cdots & \cdots & \cdots \\
    0 & 0 & 0 & \cdots & -\lambda & -a_{n - 2} \\  
    0 & 0 & 0 & \cdots & 1 & -a_{n - 1} - \lambda
\end{vmatrix}
 =$ (теорема Лапласа по 1 строке) $= 
 (-1)^{n + 1}(-a_0 - a_1 \lambda - ... - a_{n - 1} \lambda^{n - 1} - \lambda^n)
 \begin{vmatrix}
    1 & -\lambda & \cdots & 0 \\
     & \ddots & \ddots & \vdots \\
     & & \ddots & -\lambda \\
     & & & 1
 \end{vmatrix}
 = (-1)^{n + 2}f(\lambda) = (-1)^nf(\lambda)$ \\ $\blacksquare$ \\ \\ \\ 



% 22 задача
\subsection*{Задача 22:}
\addcontentsline{toc}{subsection}{Задача 22:}
\noindent Преобразование $A \rightarrow PAP^{-1}$ будем называть элементарным преобразованием подобия, если
матрица $P$ является либо матрицей перестановки, либо матрицей вида $I + \gamma E_{kl}$ , где матрица
$E_{kl}$ отличается от нулевой матрицы только одним элементом — единицей в позиции $(k,l)$ при
$k \neq l$, а число $\gamma$ может быть произвольным и выбираетcя в зависимости от ситуации. Докажите,
что матрица порядка $n$ с помощью $O(n^2)$ элементарных преобразований подобия приводится к
верхнему почти треугольному виду. \\ \\
% Решение 22 задачи
\textbf{\underline{Решение:}} \\
Пусть наша исходная матрица имеет вид 
$A = 
\begin{pmatrix}
    a_{11} & \cdots & a_{1n} \\
    \vdots & \ddots & \vdots \\
    a_{n1} & \cdots & a_{nn} 
\end{pmatrix}
$ \\ \\
На первом шаге хотим занулить все элементы на позициях $(i, 1)$ $\forall i \geq 3$. \\ \\
Если $a_{i1} \neq 0$ для некоторого $j \geq 3$, то преобразованием подобия перейдем к матрице
$$P_1AP_1^{-1} = 
\begin{pmatrix}
    a_{11} & \cdots & \cdots & \cdots & \cdots & a_{1n} \\
    a_{j1} & \ddots & \cdots & \cdots & \cdots & a_{jn} \\
    \vdots & \cdots & \ddots & \cdots & \cdots & \vdots \\
    a_{21} & \cdots & \cdots & \ddots & \cdots & a_{2n} \\
    \vdots & \cdots & \cdots & \cdots & \ddots & \vdots \\
    a_{n1} & \cdots & \cdots & \cdots & \cdots & a_{nn} 
\end{pmatrix}
$$ - здесь матрица $P$ - матрица перестановки, меняющая местами строки с номерами $2, i \Rightarrow P^{-1}$ 
поменяет местами столбцы с номерами $2, i$. \\
Теперь из каждой строки с номером $i \geq 3$ вычтем 2-ю строку, домноженную на $\frac{a_{i1}}{a_{j1}} \Rightarrow$ 
получим матрицу 
$$P_{n-1}...P_1AP_1^{-1}...P_{n - 1}^{-1} = 
\begin{pmatrix}
    a_{11} & \cdots & \cdots & \cdots & \times \\
    a_{j1} & \ddots & \cdots & \cdots &  \times \\
    0 & \cdots & \ddots & \cdots & \times \\
    \vdots & \cdots & \cdots & \ddots & \vdots \\
    0 & \cdots & \cdots & \cdots & \times
\end{pmatrix}
$$ - для того, чтобы произошло именно преобразование подобия, мы домножали справа на $P_i^{-1} \Rightarrow$ 
к $i$ столбцу прибавляли второй с тем же множителем (но это никак не повлияло на зануление первого столбца). \\
Таким образом за $n - 1$ преобразований подобия перешли к матрице, указанной выше. Теперь рассмотрим у этой матрицы подматрицу 
размеров $(n - 1) \times (n - 1)$, расположенную в правом нижнем углу. \\
К ней можно применить предположение индукции (индукция по порядку матрицы, база - $n = 1$ - уже является верхней почти треугольной). 
Т.е приведем исходную матрицу к верхней почти треугольной за $O((n - 1)^2) + (n - 1) = O(n^2)$ преобразований подобия. \\
Для подматрицы нужные нам матрицы преобразований будут иметь вид
$$P = 
\begin{pmatrix}
    1 & 0 & \cdots & 0 \\
    0 & \times & \cdots & \times \\
    \vdots & \cdots & \ddots & \vdots \\
    0 & \times & \cdots & \times
\end{pmatrix}
$$ \\
$\blacksquare$ \\ \\ \\


% 23 задача
\subsection*{Задача 23:}
\addcontentsline{toc}{subsection}{Задача 23:}
\noindent Докажите, что алгебраическая кратность собственного значения не меньше его геометрической кратности. \\ \\
% Решение 23 задачи
\textbf{\underline{Решение:}} \\
Пусть $A: V \rightarrow V$ - линейный оператор над полем $\mathbb{P}$, $dimV = n$. \\
Пусть $\lambda_i$ - собственное значение $A$. Рассмотрим произвольный базис, в котором первые 
$l_i$ векторов принадлежат собственному пространству для $\lambda_i$ (т.е $ker(A - \lambda_i I)$).
В этом базисе матрица оператора имеет вид:
$
\begin{pmatrix}
    \lambda_i I_{l_i} & A_{12} \\
    0 & A_{22}
\end{pmatrix} \Rightarrow |A - \lambda I_n| = 
(\lambda_i - \lambda)^{l_i} |A_{22} - \lambda I_{n - l_i}| \Rightarrow$ алгебраическая кратность $\lambda_i$ не меньше $l_i$. \\ $\blacksquare$ \\ \\ \\


% 24 задача
\subsection*{Задача 24:}
\addcontentsline{toc}{subsection}{Задача 24:}
\noindent Докажите, что собственные векторы для попарно различных собственных значений линейно независимы. \\ \\
% Решение 24 задачи
\textbf{\underline{Решение:}} \\
Пусть $A: V \rightarrow V$ - линейный оператор над полем $\mathbb{P}$, $\lambda_1, ... , \lambda_m$ - попарно различные собственные значения $A$ 
и $x_1, ... , x_m$ - соответствующие собственные векторы. \\
Рассмотрим их линейную комбинацию $\alpha_1 x_1 + ... + \alpha_m x_m = 0$. Умножим это равенство на $\lambda_m$, 
а также подействуем на обе его стороны линейным оператором $A$. В результате получим два равенства: \\
$$
\begin{cases}
    \alpha_1 \lambda_1 x_1 + ... + \alpha_m \lambda_m x_m = 0 \\
    \alpha_1 \lambda_m x_1 + ... + \alpha_m \lambda_m x_m = 0
\end{cases}
$$
Вычитаем из первого второе
$$\alpha_1 (\lambda_1 - \lambda_m)x_1 + ... + \alpha_{m - 1} (\lambda_{m - 1} - \lambda_m)x_{m - 1} = 0$$
То есть если $x_1, ... , x_{m - 1}$ - линейно независимы, то $\alpha_1 = ... = \alpha_{m - 1} = 0 \Rightarrow
\alpha_m x_m = 0 \Rightarrow \alpha_m = 0$ (так как $x_m \neq 0$) \\
Далее применяем индукцию (база - 1 вектор очевидно линейно независим сам с собой). \\ $\blacksquare$ \\ \\ \\


% 25 задача
\subsection*{Задача 25:}
\addcontentsline{toc}{subsection}{Задача 25:}
\noindent Докажите, что если матрица порядка $n$ имеет $n$ попарно различных собственных значений, то она диагонализуема. \\ \\
% Решение 25 задачи
\textbf{\underline{Решение:}} \\
Пусть $A \in \mathbb{R}^{n \times n}$ - матрица с $n$ попарно различными собственными значениями $\lambda_1, ... , \lambda_n$ и 
$x_1, ... , x_n$ - соответствующие собственные векторы. \\
Из предыдущей задачи знаем, что векторы $x_1, ... , x_n$ - линейно независимы $\Rightarrow$ они образуют
базис $\mathbb{R}^n$. \\
Составим из них матрицу $X = (x_1, ... , x_n)$.
Тогда $AX = X
\begin{pmatrix}
    \lambda_1 & & \\
     & \ddots & \\
     & & \lambda_n
\end{pmatrix}
$ (так как $Ax_i = \lambda_i x_i$) $\Rightarrow XAX^{-1} =
\begin{pmatrix}
    \lambda_1 & & \\
     & \ddots & \\
     & & \lambda_n
\end{pmatrix}$ - диагональная матрица $\Rightarrow$ $A$ - диагонализуема. \\ $\blacksquare$ \\ \\ \\


% 26 задача
\subsection*{Задача 26:}
\addcontentsline{toc}{subsection}{Задача 26:}
\noindent Докажите, что для нильпотентности линейного оператора необходимо и достаточно, чтобы он был квазискалярным с единственным собственным значением, равным нулю. \\ \\
% Решение 26 задачи
\textbf{\underline{Решение:}} \\
1. $\Rightarrow$: \\
Пусть $A: V \rightarrow V$ - нильпотентный линейный оператор, то есть $\exists m$: $A^m = 0$ \\
Тогда $\forall$ собственного вектора $x$: $Ax = \lambda x \Rightarrow A^m x = \lambda^m x \Rightarrow
\lambda^m x = 0 \Rightarrow \lambda = 0 \Rightarrow A$ - квазискалярный с единственным собственным значением $\lambda = 0$ \\
2. $\Leftarrow$: \\
Пусть $A: V \rightarrow V$ - квазискалярный линейный оператор с собственным значением $\lambda = 0$ \\
Тогда его характеристический многочлен имеет вид $f(\lambda) = (-\lambda)^n$ \\
По теореме Кэли-Гамильтона $f(A) = 0 \Rightarrow (-1)^n A^n = 0 \Rightarrow A^n = 0 \Rightarrow A$ - нильпотентный. \\
$\blacksquare$ \\ \\ \\


% 27 задача
\subsection*{Задача 27:}
\addcontentsline{toc}{subsection}{Задача 27:}
\noindent Докажите, что линейный оператор $A$ является квазискалярным с единственным собственным
значением $\lambda$ тогда и только тогда, когда сдвинутый оператор $A - \lambda I$ является нильпотентным. \\ \\
% Решение 27 задачи
\textbf{\underline{Решение:}} \\
1. $\Rightarrow$: \\
Пусть $A: V \rightarrow V$ - квазискалярный линейный оператор с собственным значением $\lambda$ \\
Тогда сдвинутый оператор $A - \lambda I$ - нильпотентный. Докажем: \\
$\forall$ собственного вектора оператора $A$ имеем: $(A - \lambda I)x = Ax - \lambda x = 0 \Rightarrow x$ - 
собственный для сдвинутого оператора с собственным значением  $\lambda = 0$ \\
Если $\exists x: (A - \lambda I)x = \alpha x, \alpha \neq 0$, то $Ax = (\lambda + \alpha)x \Rightarrow A$ - не квазискалярный
$\Rightarrow$ противоречение $\Rightarrow A - \lambda I$ - квазискалярный с собственным значением $\lambda = 0 \Rightarrow$ 
он нильпотентный (по предыдущей задаче). \\
2. $\Leftarrow$: \\
$A - \lambda I$ - нильпотентный $\Rightarrow$ он квазискалярный с единственным собственным значением $\lambda = 0$ 
(по предыдущей задаче). \\
Тогда $\forall$ собственного вектора оператора $A - \lambda I$ имеем: $(A - \lambda I)x = Ax - \lambda x = 0 \Rightarrow Ax = \lambda x \Rightarrow x$ - 
собственный для $A$ с собственным значением  $\lambda$ \\
Если $\exists x: Ax = \alpha x, \alpha \neq \lambda$, то $(A - \lambda I)x = (\alpha - \lambda)x \Rightarrow A -\lambda I$ - не квазискалярный
$\Rightarrow$ противоречение $\Rightarrow A$ - квазискалярный с собственным значением $\lambda$ \\
$\blacksquare$ \\ \\ \\


% 28 задача
\subsection*{Задача 28:}
\addcontentsline{toc}{subsection}{Задача 28:}
\noindent Докажите, что любой вырожденный линейный оператор либо является нильпотентным, либо
расщепляется в прямую сумму нильпотентного и обратимого операторов. \\ \\
% Решение 28 задачи
\textbf{\underline{Решение:}} \\
Пусть $A: V \rightarrow V$ - вырожденный линейный оператор, $dimV = n$ \\
По теореме об обратимом операторе - $kerA \neq \{0\}$ (то есть ядро оператора $A$ нетривиально). 
Обозначим $N_k = kerA^k, M_k = imA^k$ \\ \\
Рассмотрим $N_k$ и $N_{k + 1}$: \\
$x \in N_k \Rightarrow A^k x = 0 \Rightarrow A^{k + 1} x = A(A^k x) = A \cdot 0 = 0 \Rightarrow x \in N_{k + 1} 
\Rightarrow N_k \subseteq N_{k + 1}$ \\
Для $M_k$ и $M_{k + 1}$: \\
$x \in M_{k + 1} \Rightarrow x = A^{k + 1} y = A^k (Ay) \Rightarrow x \in M_k \Rightarrow M_{k + 1} \subseteq M_k$ \\ \\
Так как $V$ - конечномерное, то $\exists m \in \mathbb{N}$ такое, что $N_1 \subset ... \subset N_{m - 1} = N_m$ 
(в силу конечномерности ядра не могут расширяться бесконечно). \\
По теореме о ранге и дефекте:
$$
\begin{cases}
    dimN_{m - 1} + dimM_{m - 1} = n \\
    dimN_m + dimM_m = n \\
    dimN_m = dimN_{m - 1}
\end{cases}
\Rightarrow dimM_{m - 1} = dimM_m \Rightarrow M_{m - 1} = M_m
$$ \\
То есть
$$
\begin{cases}
    N_1 \subset N_2 \subset ... \subset N_{m - 1} = N_m \\
    M_1 \supset M_2 \supset ... \supset M_{m - 1} = M_m
\end{cases}
$$ \\ \\
Покажем теперь, что $N_{m + k} = N_{m - 1} \forall k \in \mathbb{N}$.  Достаточно показать для $N_{m + 1}$: \\
$x \in N_{m + 1} \Rightarrow A^{m + 1} x = 0 \Rightarrow A^m (Ax) = 0 \Rightarrow Ax \in N_m = N_{m - 1} \Rightarrow 
Ax \in N_{m - 1} \Rightarrow A^{m - 1} (Ax) = 0 \Rightarrow A^m x = 0 \Rightarrow x \in N_m = N_{m - 1} \Rightarrow 
x \in N_{m - 1} \Rightarrow N_{m - 1} =N_{m + 1}$ \\
Для всех $k \geq 2$ - аналогично. \\
По теореме о ранге и дефекте аналогично получаем, что $M_{m + k} = M_{m - 1} \forall k \in \mathbb{N}$ \\ \\
Теперь задача разбивается на 2 случая: \\
1. $N_{m - 1} = V \Rightarrow A^{m - 1} = 0 \Rightarrow A$ - нильпотентный. \\
2. $N_{m - 1} \neq V \Rightarrow N_{m - 1}, M_{m - 1}$ - не пустые ($N_{m - 1}$ - не пустое, так как 
$A$ - вырожденный $\Rightarrow A^{m - 1}$ - вырожденный; $M_{m - 1}$ - не пустое по теореме о ранге и дефекте). \\  \\
Докажем, что $V = N_{m - 1} \oplus M_{m - 1}$: \\
$dimN_{m - 1} + dimM_{m - 1} = dim V$ по теореме о ранге и дефекте. Кроме того, \\
$x \in N_{m - 1} \cap M_{m - 1} \Rightarrow A^{m - 1} x = 0, x = A^{m - 1} y \Rightarrow 
A^{2(m - 1)} y = 0 \Rightarrow y \in N_{2(m - 1)} = N_{m - 1} \Rightarrow x = A^{m - 1} y = 0 \Rightarrow 
N_{m - 1} \cap M_{m - 1} = \{0\}$ \\
Из этого следует, что $V = N_{m - 1} \oplus M_{m - 1}$ (более подробное пояснение во втором пункте задачи 10 из этого же раздела). \\ \\
Также $N_{m - 1}, M_{m - 1}$ - инвариантные для $A$: \\
$x \in N_{m - 1} \Rightarrow A^{m - 1} x = 0 \Rightarrow A^{m - 1} (Ax) = A(A^{m - 1} x) = A \cdot 0 = 0 \Rightarrow Ax \in N_{m - 1}$ \\
$x \in M_{m - 1} \Rightarrow x = A^{m - 1} y \Rightarrow Ax = A^m y = A^{m - 1} (Ay) \Rightarrow Ax \in M_{m - 1}$ \\ \\
Теперь: \\
$A^{m - 1} x = 0$ $\forall x \in N_{m - 1} \Rightarrow A|N_{m - 1}$ - нильпотентный на $N_{m - 1}$ \\
$\forall x \in M_{m - 1}: Ax = A^m y \neq 0$, если $x \neq 0$, так как иначе: \\
$A^m y = 0 \Rightarrow y \in N_m = N_{m - 1} \Rightarrow x = A^{m - 1} y = 0$ - противоречие. \\
То есть единственный вектор, который переводится в $0$ - нулевой $\Rightarrow ker(A|M_{m - 1}) = \{0\} \Rightarrow 
A|M_{m - 1}$ - обратимый (по теореме об обратимом операторе) \\
$\blacksquare$ \\ \\ \\


% 29 задача
\subsection*{Задача 29:}
\addcontentsline{toc}{subsection}{Задача 29:}
\noindent Пусть линейный оператор действует на конечномерном пространстве над полем, которое содержит 
все корни его характеристического многочлена. Докажите, что он расщепляется в прямую
сумму своих сужений на корневые пространства своих попарно различных собственных значений. \\ \\
% Решение 29 задачи
\textbf{\underline{Решение:}} \\
1. Для начала докажем вспомогательную теорему: \\ \\
Пусть линейный оператор имеет собственное значение $\lambda_i$ алгебраической кратности $k_i$. 
Тогда либо он является квазискалярным, либо расщепляется в прямую сумму квазискалярного оператора, 
действующего на подпространстве размерности $k_i$, и оператора, для которого $\lambda_i$ не является собственным значением. \\ \\
\underline{Доказательство:} \\
Рассмотрим сдвинутый оператор $B = A - \lambda_i I$ - вырожденный, так как имеет собственное значение 
$\lambda = 0$ алгебраической кратности $k_i \Rightarrow kerB \neq \{0\}$ \\
Из предыдущей задачи имеем: \\
1. $B$ - нильпотентный $\Rightarrow A$ - квазискалярный \\
2. $B = C \oplus D$, где $C$ - нильпотентный, $D$ - обратимый. То есть (по определению прямой суммы операторов) 
$V = L \oplus M$, $C = B|L = (A - \lambda_i I)|L, D = B|M = (A - \lambda_i I)|M$, $L = kerB^m, M = imB^m, m$ - степень, начиная с которой 
ядра и образы перестают вкладываться друг в друга - по построению из предыдущей задачи. \\ \\
$D$ - обратимый $\Rightarrow \lambda = 0$ не является собственным значением $D \Rightarrow 
\lambda = \lambda_i$ не является собственным значением $A|M$ ($A = D + \lambda_i I$) \\ \\
Знаем, что $\lambda = 0$ - собственное значение $B$ алгебраической кратности $k_i$, характеристический многочлен $B$ 
является произведением характеристических многочленов $C$ и $D$, и $\lambda = 0$ не является собственным 
значением $D \Rightarrow dimL = k_i$, так как корень $\lambda = 0$ кратности $k_i$ полностью дает характеристический 
многочлен $C$, а в силу нильпотентности оператора $C$ других корней у него нет, а значит $dimL$ в точности равно $k_i$. \\ \\
То есть $A|L = (C + \lambda_i I)|L$ - квазискалярный и $dimL = k_i$ \\ \\
2. Теперь перейдем непосредственно к решению задачи: \\
Пусть $A: V \rightarrow V$ - линейный оператор, имеющий $s$ попарно различных собственных значений $\lambda_1, ... , \lambda_s$ кратностей 
$k_1, ... , k_s$. Ведем индукцию по $s$: \\
\begin{itemize}
    \item $s = 1 \Rightarrow A$ - квазискалярный $\Rightarrow (A - \lambda I)$ - нильпотентный 
    $\Rightarrow ker(A - \lambda I)^n = V$ - верно
    \item $s \geq 2 \Rightarrow A = A|K_1 \oplus A|W$, где $W$ - инвариантное пространство, на котором $A|W$ 
    не имеет собственного значения $\lambda_1$ (по вспомогательной теореме), $K_1$ - корневое для $\lambda_1$ 
    (так как у нас по построению из вспомогательной теоремы $K_1 = ker(A - \lambda_1 I)^{k_1}$, а 
    $m \leq k_1$, так как иначе $dim(ker(A - \lambda_1 I)^m) > k_1$ - противоречие с доказанным во вспомогательной теореме). \\
    По предположению индукции $W = W_2 \oplus ... \oplus W_s$, где $W_i$ - корневое пространство для $A|W$. 
    Так как кратности $k_2, ... , k_s$ у $A$ и $A|W$ одинаковые, то $W_i = K_i$ (так как $W_i \subseteq K_i$ и при 
    этом $dimW_i = dimK_i = k_i$). \\
    Значит $V = K_1 \oplus W = K_1 \oplus W_2 \oplus ... \oplus W_s = K_1 \oplus K_2 \oplus ... \oplus K_s$ \\
\end{itemize}
$\blacksquare$ \\ \\ \\


% 30 задача
\subsection*{Задача 30:}
\addcontentsline{toc}{subsection}{Задача 30:}
\noindent Матрица $A$ порядка $n$ имеет попарно различные собственные значения $\lambda_1, ... , \lambda_n$ 
и соответствующие им собственные векторы $v_1, ... , v_n$. Найти собственные значения и собственные векторы линейного оператора 
$B: \mathbb{C}^{n \times n} \rightarrow \mathbb{C}^{n \times n}, BX = A^3 X A^4$ \\ \\
% Решение 30 задачи
\textbf{\underline{Решение:}} \\
Так как матрица $A$ имеет попарно различные собственные значения, то она диагонализуема (задача 25 из этого раздела). \\
Значит $A = V\Lambda V^{-1}, V = [v_1, ... , v_n], \Lambda = diag(\lambda_1, ... , \lambda_n)$ \\ \\
Тогда $BX = A^3 X A^4 = V \Lambda^3 V^{-1} XV\Lambda^4 V^{-1}$ \\
Пусть $X$ - собственный вектор оператора $B \Rightarrow V \Lambda^3 V^{-1} XV\Lambda^4 V^{-1} = \lambda X \iff$ 
$\Lambda^3 (V^{-1} XV)\Lambda^4 = \lambda (V^{-1} XV)$ \\
Таким образом, $X$ - собственный вектор оператора $B$ $\iff$ $Y = V^{-1} XV$ - собственный вектор оператора 
$C: \mathbb{C}^{n \times n} \rightarrow \mathbb{C}^{n \times n}, CX = \Lambda^3 X \Lambda^4$ \\ \\
Рассмотрим $Y = E_{ij}$ - матрица, в которой на позиции $(i, j)$ стоит $1$, а на всех остальных - $0$. Заметим, что 
$CY = \Lambda^3 Y \Lambda^4 = \lambda_i^3 \lambda_j^4 E_{ij} \Rightarrow Y$ - собственный вектор для $C$. \\
То есть нашли $n^2$ линейно независимых собственных векторов оператора $C \Rightarrow 
X_{ij} = VE_{ij}V^{-1}$ - собственные для $B$, причем также линейно независимые: \\
$\alpha_{11}X_{11} + ... + \alpha_{nn}X_{nn} = 0 \Rightarrow V(\alpha_{11}E_{11} + ... + \alpha_{nn}E_{nn})V^{-1} = 0 
\Rightarrow \alpha_{11}E_{11} + ... + \alpha_{nn}E_{nn} = 0 \Rightarrow \alpha_{11} = ... = \alpha_{nn} = 0$ \\
То есть нашли все собственные векторы $B$: $X_{ij} = VE_{ij}V^{-1}$ для собственного значения $\lambda_i^3 \lambda_j^4$ \\
$\blacksquare$ \\ \\ \\


% 31 задача
\subsection*{Задача 31:}
\addcontentsline{toc}{subsection}{Задача 31:}
\noindent Докажите, что минимальное инвариантное относительно оператора $A$ подпространство $M(A, x)$,
содержащее заданный ненулевой вектор $x$, совпадает с пространством Крылова $L_k(A, x)$, содержащим 
вектор $A^k x$. Его размерность равна минимальному значению $k$, при котором $A^k x \in L_k(A, x)$. \\ \\
% Решение 31 задачи
\textbf{\underline{Решение:}} \\
Пусть векторы $x, Ax, ... , A^{k - 1} x$ - линейно независимы, а вектор $A^k x$ выражается в виде их линейной комбинации. 
Понятно, что в таком случае $dim(L_k(A, x)) = k$ (так как это по определению линейная оболочка $k$ линейно независимых 
векторов). \\ \\
Пусть $A^k x = \alpha_1 x + ... + \alpha_k A^{k - 1} x$. Тогда $\forall y \in L_k(A, x): 
y = \beta_1 x + ... + \beta_k A^{k - 1} x$ имеем $Ay = \beta_1 Ax + ... + \beta_k A^k x = 
\beta_1 Ax + ... + \beta_{k - 1} A^{k - 1} x + \beta_k(\alpha_1 x + ... + \alpha_k A^{k - 1} x) = 
\beta_k \alpha_1 x + (\beta_1 + \beta_k \alpha_2) Ax + ... + (\beta_{k - 1} + \beta_k \alpha_k) A^{k - 1} x \in L_k(A, x) \Rightarrow 
L_k(A, x)$ - инвариантное для $A$. \\ \\
Очевидно, что любое инвариантное пространство, содержащее вектор $x$, должно содержать векторы $Ax, ... , 
A^{k - 1} x, ...$ $\Rightarrow$ оно содержит $L_k(A, x) \Rightarrow M(A, x) = L_k(A, x)$ и $dim(L_k(A, x)) = k$ \\ \\
Также заметим, что $\forall n \in \mathbb{N}: L_{k + n}(A, x) = L_k(A, x)$, так как $A^{k + n} x \in L_k(A, x)$ в силу его инвариантности. \\
То есть $M(A, x) = L_m(A, x)$ $\forall m \geq k$ и при этом $dim(L_m(A, x)) = k$. \\ \\
\textit{В конце я показал, что в качестве минимального пространства мы можем взять любое крыловское, для которого выполнено 
$A^m x \in L_m(A, x)$, однако его размерность будет равна минимальному значению $m$, при котором это выполняется} \\
$\blacksquare$ \\ \\ \\ 


% 32 задача
\subsection*{Задача 32:}
\addcontentsline{toc}{subsection}{Задача 32:}
\noindent Докажите, что минимальное инвариантное относительно оператора $A$ подпространство $M(A, x)$,
содержащее заданный вектор $x \neq 0$, нерасщепляемо в том и только том случае, когда сужение
оператора $A$ на нем квазискалярно. \\ \\
% Решение 32 задачи
\textbf{\underline{Решение:}} \\
1. $\Rightarrow$: \\
Предположим, что сужение не квазискалярно $\Rightarrow$ пространство можно расщепить в прямую сумму корневых подпространств 
(задача 29 из этого раздела) $\Rightarrow$ противоречие с минимальностью пространства $M(A, x)$. \\
То есть сужение оператора на нем квазискалярно. \\ \\
2. $\Leftarrow$: \\
По предыдущей задаче знаем, что $M(A, x) = L_k(A, x) := L(x, Ax, ... , A^{k - 1} x)$, причем $A^k x \in L_k(A, x)$ и 
$dimL_k(A, x) = k$ \\
Пусть единственное собственное значение сужения $A|M(A, x)$ равно $\lambda$ \\
Рассмотрим оператор $B := A - \lambda I$ - для него $B|M(A, x)$ - нильпотентный. \\
Докажем, что $L_k(B, x) = L_k(A, x)$: \\
\begin{itemize}
    \item Пусть $y \in L_k(B, x) \Rightarrow y = \alpha_1 x + ... + \alpha_k B^{k - 1}x = 
    \alpha_1 x + ... + \alpha_k (A - \lambda I)^{k - 1}x = \beta_1 x + ... + \beta_k A^{k - 1}x$ (так как единичный
    оператор коммутирует с любым, то можем раскрыть степени по биному и сгруппировать коэффициенты при одинаковых степенях $A \Rightarrow$ 
    получим, что $y \in L_k(A, x)$) $\Rightarrow L_k(B, x) \subseteq L_k(A, x)$ \\
    \item Пусть $y \in L_k(A, x) \Rightarrow$ действуем аналогично, только теперь $A = B + \lambda I \Rightarrow 
    y \in L_k(B, x) \Rightarrow L_k(A, x) \subseteq L_k(B, x)$
\end{itemize}
Доказали двухстороннюю вложенность $\Rightarrow L_k(A, x) = L_k(B, x) \Rightarrow dimL_k(B, x) = k \Rightarrow 
x, Bx, ... , B^{k - 1}x$ - линейно независимые \\ \\
Пусть $L \subseteq M(A, x)$ - ненулевое инвариантное пространство для $B$ (а значит и для $A$). \\
Возьмем $z \in L, z \neq 0 \Rightarrow z = \sum_{j = i}^{k - 1} \alpha_j B^j x$, где $\alpha_1 = ... = \alpha_{i - 1} = 0, 
\alpha_i \neq 0$ \\
Так как $B$ - нильпотентный и $dimM(A, x) = k$, то $B^k x = 0$ $\forall x \in M(A, x)$ (например потому, что по теореме 
Кэли-Гамильтона имеем $(B|M(A, x))^k = 0$). \\
Тогда $B^{k - i - 1}z = \alpha_i B^{k - 1}x \in L$, так как $L$ - инвариантное $\Rightarrow B^{k - 1}x \in L$ \\
В силу произвольного выбора $L$ получаем, что все инвариантные пространства имеют общий вектор $B^{k - 1}x \Rightarrow 
M(A, x)$ нельзя расщепить в прямую сумму инвариантных подпространств. \\
$\blacksquare$ \\ \\ \\


% 33 задача
\subsection*{Задача 33:}
\addcontentsline{toc}{subsection}{Задача 33:}
\noindent Докажите, что если $B$ — нильпотентный линейный оператор, векторы $x, Bx, ... , B^{k - 1} x$ 
ненулевые и $B^k x = 0$, то векторы $x, Bx, ... , B^{k - 1} x$ линейно независимы. \\ \\
% Решение 33 задачи
\textbf{\underline{Решение:}} \\
Рассмотрим линейную комбинацию векторов $x, Bx, ... , B^{k - 1}x$, равную $0$:
$$\alpha_1 x + \alpha_2 Bx + ... + \alpha_k B^{k - 1}x = 0$$
Подействуем на обе части равенства оператором  $B^{k - 1}$: 
$$\alpha_1 B^{k - 1}x = 0 \Rightarrow \alpha_1 = 0$$
То есть получили, что 
$$\alpha_2 Bx + ... + \alpha_k B^{k - 1}x = 0$$
Теперь подействуем на обе части равенства оператором $B^{k - 2} \Rightarrow$ получим $\alpha_2 = 0$. Действуя аналогично, 
получим, что $\alpha_1 = \alpha_2 = ... = \alpha_k = 0 \Rightarrow x, Bx, ... , B^{k - 1}x$ - линейно независимы. \\
$\blacksquare$ \\ \\ \\


% 34 задача
\subsection*{Задача 34:}
\addcontentsline{toc}{subsection}{Задача 34:}
\noindent Линейный оператор $A$ называется нильпотентным на векторе $x \neq 0$, если существует 
натуральное число $k$, для которого $A^k x = 0$. Минимальное такое $k$ называется индексом 
нильпотентности оператора $A$ на векторе $x$. Пусть $A$ — линейный оператор и $k_1, ... , k_t$ — его индексы 
нильпотентности на ненулевых векторах $x_1, ... , x_t$. Докажите, что для линейной независимости составной системы векторов Крылова 
$x_1, Ax_1, ... , A^{k_1 - 1} x_1, ... , x_t , Ax_t, ... , A^{k_t - 1} x_t$ 
необходима и достаточна линейная независимость векторов $A^{k_1 - 1} x_1, ... , A^{k_t - 1} x_t$. \\ \\
% Решение 34 задачи
\textbf{\underline{Решение:}} \\
1. $\Rightarrow$: \\
Из линейной независимости системы векторов очевидно следует линейная независимость любой ее подсистемы. \\ \\
2. $\Leftarrow$: \\
Пусть 
$$
\begin{cases}
    x_1, Ax_1, ... , A^{k_1 - 1} x_1 \\
    \cdots \cdots \cdots \cdots \\
    x_t , Ax_t, ... , A^{k_t - 1} x_t
\end{cases}
\text{ - данная нам система векторов}$$ $$k_1, ... , k_t \text{- индексы нильпотентности}$$ \\
Обозначим $k = max\{k_1, ... , k_t\}$ и будем вести индукцию по $k$.
\begin{itemize}
    \item $k = 1 \Rightarrow$ вся система векторов линейно независима по условию $\Rightarrow$ верно.
    \item $k \geq 2 \Rightarrow$ рассмотрим линейную комбинацию 
    $$\alpha_{11}x_1 + ... + \alpha_{1k_1}A^{k_1 - 1}x_1 + ... + \alpha{t1}x_t + ... + \alpha_{tk_t}A^{k_t - 1}x_t = 0 \text{ $(*)$}$$ \\
    Подействуем на обе части равенства оператором $A^{k - 1} \Rightarrow$ получим
    $$\alpha_{i_11}A^{k - 1}x_{i_1} + ... + \alpha_{i_r_1 }A^{k - 1}x_{i_r} = 0$$ Здесь $x_{i_1}, ... , x_{i_r}$ - 
    векторы, для которых $k_{i_1} = ... = k_{i_r} = k$ \\
    Значит, $\alpha_{i_11} = ... = \alpha_{i_r1} = 0$, так как $A^{k - 1}x_{i_1}, ... , A^{k - 1}x_{i_r}$ - линейно 
    независимы по условию как подсистема $A^{k_1 - 1} x_1, ... , A^{k_t - 1} x_t$ \\
    Не ограничивая общности, будем считать, что $x_{i_1} = x1, ... , x_{i_r} = x_r$ \\
    Тогда обозначим $y_1 = Ax_1, ... , y_r = Ax_r$ и рассмотрим систему векторов 
    $$
    \begin{cases}
        y_1, Ay_1, ... , A^{k_1 - 2} y_1 \\
        \cdots \cdots \cdots \cdots \\
        y_r, Ay_r, ... , A^{k_r - 2}y_r \\
        \cdots \cdots \cdots \cdots \\
        x_t , Ax_t, ... , A^{k_t - 1} x_t
    \end{cases}
    $$ \\
    Для этой системы максимальный индекс нильпотентности уменьшился, так как мы исключили из исходной системы векторы, у которых он был 
    равен $k \Rightarrow$ данная система линейно независима по предположению индукции $\Rightarrow$ все 
    остальные коэффициенты в $(*)$, кроме $\alpha_{11}, ... , \alpha_{r1}$, тоже нулевые $\Rightarrow$ 
    исходная система векторов линейно независима.
\end{itemize}
$\blacksquare$ \\ \\ \\


% 35 задача
\subsection*{Задача 35:}
\addcontentsline{toc}{subsection}{Задача 35:}
\noindent Докажите, что любой нильпотентный оператор, действующий на конечномерном пространстве,
расщепляется в прямую сумму нерасщепляемых операторов, действующих на инвариантных подпространствах Крылова. \\ \\
% Решение 35 задачи
\textbf{\underline{Решение:}} \\
Пусть $A: V \rightarrow V$ - нильпотентный линейный оператор, $dimV = n$. Будем вести индукцию по $n$. 
\begin{itemize}
    \item $n = 1 \Rightarrow$ всё пространство является пространством Крылова $\Rightarrow$ верно
    \item $n \geq 2$: \\
    $A$ - нильпотентный $\Rightarrow defA > 0 \Rightarrow rgA < n$. То есть $imA$ - инвариантное пространство 
    для $A$ размерности $\leq n - 1$. \\
    Обозначим $W := imA \Rightarrow dimW \leq n - 1$ \\
    В силу индуктивного предположения:
    $$W = L_{k_1}(A, x_1) \oplus ... \oplus L_{k_t}(A, x_t) \text{ - расщепление на инвариантные пространства Крылова}$$
    Общее число векторов в системе 
    $$
    \begin{cases}
        x_1, Ax_1, ... , A^{k_1 - 1} x_1 \\
        \cdots \cdots \cdots \cdots \\
        x_t , Ax_t, ... , A^{k_t - 1} x_t
    \end{cases}
    $$
    равно $r = rgA$, так как $dimW = rgA$ \\
    Так как расщепляли образ, то каждый из векторов $x_1, ... , x_t \in imA \Rightarrow x_1 = Ay_1, ... , x_t = Ay_t$ \\
    Рассмотрим составную систему Крылова:
    $$
    \begin{cases}
        y_1, Ay_1, ... , A^{k_1} y_1 \\
        \cdots \cdots \cdots \cdots \\
        y_t , Ay_t, ... , A^{k_t} y_t
    \end{cases}
    \text{ - в ней t + r векторов}
    $$
    Векторы $A^{k_1}y_1 = A^{k_1 - 1}x_1, ... , A^{k_t}y_t = A^{k_t - 1}x_t$ линейно независимы (как подсистема базиса $W$, 
    составленного из базисов пространств Крылова) и лежат в ядре $A$, так как из нильпотентности $A$ следует 
    нильпотентность его сужений $A|L_{k_i} \Rightarrow A^{k_i}x = 0$ $\forall x \in L_{k_i}$ \\
    Если $t < defA$, то дополним $A^{k_1}y_1, ... , A^{k_t}y_t$ до базиса $kerA$ векторами $y_{t + 1}, ... , y_d$, где 
    $d = defA$ \\
    Тогда система 
    $$
    \begin{cases}
        y_1, Ay_1, ... , A^{k_1} y_1 \\
        \cdots \cdots \cdots \cdots \\
        y_t , Ay_t, ... , A^{k_t} y_t \\
        y_{t + 1} \\
        \cdots \\
        y_d
    \end{cases}
    $$
    линейно независима в силу линейной независимости последних векторов в ней (смотри предыдущую задачу) $\Rightarrow$ 
    $$V = L_{k_1 + 1}(A, y_1) \oplus ... \oplus L_{k_t + 1}(A, y_t) \oplus L_1(A, y_{t + 1}) \oplus ... \oplus L_1(A, y_d)$$
    Все пространства Крылова в этом расщеплении нерасщепляемы, так как $A$ - нильпотентный $\Rightarrow$ квазискалярный $\Rightarrow$ 
    сужения $A$ на пространства Крылова квазискалярны $\Rightarrow$ они нерасщепляемы (задача 32 из этого раздела)
\end{itemize}
$\blacksquare$ \\ \\ \\


% 36 задача
\subsection*{Задача 36:}
\addcontentsline{toc}{subsection}{Задача 36:}
\noindent Докажите, что в любом расщеплении конечномерного пространства в прямую сумму инвариантных 
подпространcтв Крылова для нильпотентного оператора $A$ число подпространств размерности $k$ равно $N_k = 2defA^k - defA^{k - 1} - defA^{k + 1}$. \\ \\
% Решение 36 задачи
\textbf{\underline{Решение:}} \\
Пусть $A: V \rightarrow V$ - нильпотентный линейный оператор, $dimV = n$ и
\begin{gather*}
    V = L_{k_1}(A, x_1) \oplus ... \oplus L_{k_s}(A, x_s)
    \text{ - произвольное расщепление $V$ в прямую сумму} \\ \text{инвариантных пространств Крылова}
\end{gather*}
Тогда базис всего пространства $V$ имеет вид 
$$x_1, ... , A^{k_1 - 1}x_1, ... , x_s, ... , A^{k_s - 1}x_s$$
Здесь $k_1, ... , k_s \text{- индексы нильпотентности $A$ на векторах $x_1, ... , x_s$}$ \\ \\
Обозначим $N_{\geq k}$ - число пространств размерности $\geq k$ и докажем, что $N_{\geq k} = defA^k - def^A{k - 1}$:
\begin{enumerate}
    \item $N_{\geq 1} = s = defA = defA^1 - defA^0$ (так как $A^{k_1 - 1}x_1, ... , A^{k_s - 1}x_s \in kerA$, то, 
    если $s < defA$, то дополнили бы до базиса ядра векторами $x_{s + 1}, ... , x_d \Rightarrow$ получили бы базиса пространства вида\\
    \underbrace{$x_1, ... , A^{k_1 - 1}x_1, ... , x_s, ... , A^{k_s - 1}x_s, x_{s + 1}, ... , x_d}_\text{последние векторы системы линейно независимы, значит вся система 
    линейно независима (задача 34 из этого раздела)} \Rightarrow$ противоречение с размерностью $V$)
    \item Заметим, что $N_{\geq k}$ - это число векторов вида $A^{k_i - l}x_i$ при $1 \leq l \leq min\{k, k_i\}, 1 \leq i \leq s$ 
    минус число векторов вида $A^{k_i - l}x_i$ при $1 \leq l \leq min\{k - 1, k_i\}, 1 \leq i \leq s$ \\
    \underline{Пояснение} \\
    Еще раз изобразим наши базисы пространств Крылова из прямой суммы:
    $$
    \begin{cases}
        L_{k_1}: x_1, ... , A^{k_1 - 1}x_1 \\
        \cdots \cdots \cdots \cdots \\
        L_{k_s}: x_s, ... , A^{k_s - 1}x_S
    \end{cases}
    $$
    Заметим, что $N_{\geq k}$ - это число пространств Крылова, у которых $k_i \geq k$. \\
    Теперь посмотрим, что такое $A^{k_i - l}x_i, 1 \leq l \leq min\{k, k_i\}$:
    \begin{itemize}
        \item При $k > k_i$ получаем $min\{k, k_i\} = k_i \Rightarrow l$ пробегает все значения от $1$ до $k_i \Rightarrow$ 
        получим все $k_i$ векторов из пространства $L_{k_i}$
        \item При $k_i \geq k$ получаем $min\{k, k_i\} = k \Rightarrow l$ пробегает все значения от $1$ до $k \Rightarrow$ 
        получим последние $k$ векторов из пространства $L_{k_i}$
    \end{itemize}
    Что такое $A^{k_i - l}x_i, 1 \leq l \leq min\{k - 1, k_i\}$:
    \begin{itemize}
        \item При $k > k_i$ получаем $min\{k - 1, k_i\} = k_i \Rightarrow l$ пробегает все значения от $1$ до $k_i \Rightarrow$ 
        получим все $k_i$ векторов из пространства $L_{k_i}$
        \item При $k_i \geq k$ получаем $min\{k - 1, k_i\} = k - 1 \Rightarrow l$ пробегает все значения от $1$ до $k \Rightarrow$ 
        получим последние $k - 1$ векторов из пространства $L_{k_i}$
    \end{itemize}
    То есть, когда мы вычтем, то получим, что для пространств, где $k > k_i$ будет разность $k_i - k_i = 0$ \\
    Для пространств, где $k_i \geq k$ будет разность $k - (k - 1) = 1$ \\
    Значит, это число действительно равно $N_{\geq k}$
    \item Заметим, что все векторы вида $A^{k_i - l}x_i, 1 \leq l \leq min\{k, k_i\}$ лежат в ядре $A^k$ (так как 
    $k + k_i - l \geq k_i \Rightarrow A^k (A^{k_i - l})x_i = 0$, так как $k_i$ - индекс нильпотентности $A$ на $x_i$) \\
    Кроме того, никакие другие векторы из базиса $V$ не лежат в ядре $A^k$ (так как при $k \ge k_i$ получаем, что $A^{k_i - l}x_i, 1 \leq l \leq min\{k, k_i\}$
     - это все векторы из $L_{k_i}$, а при $k_i \geq k$ - если $k \textless l \leq k_i$, то $k + k_i - l \textless 
     k + k_i - k = k_i \Rightarrow A^k(A^{k_i - l}x_i) \neq 0$) \\
    При этом размерность образа $A^k$ равна числу векторов базиса $V$, не лежащих в ядре. Покажем это: \\
    Возьмем произвольный $x \in V$ и распишем его по базису: \\
    $$x = \alpha_1 x_1 + ... + \alpha_{k_1}(A^{k_1 - 1}x_1) + ... + \alpha_n (A^{k_s - 1}x_s)$$
    Тогда $$A^k x = \alpha_{i_1}(A^k y_1) + ... + \alpha_{i_r}(A^k y_r)$$
    Здесь $y_1, ... , y_r$ - все векторы из базиса $V$, кроме тех, что лежат в ядре $A^k$ (так как те, что в ядре, занулятся), причем 
    $y_1 = A^{m_{i_1}}x_{i_1}, ... , y_r = A^{m_{i_r}}x_{i_r}$ (так как все векторы из базиса $V$ имеют такой вид) 
    и $A^k y_j$ - тоже один из векторов базиса $V$ $\forall 1 \leq j \leq r$ (так как $y_j$ не в ядре $A^k \Rightarrow$ 
    $k + m_{i_j} \textless k_{i_j} \Rightarrow A^k y_j = A^{k + m_{i_j}}x_{i_j}$ - вектор из базиса $V$) $\Rightarrow imA^k = L(A^k y_1, ... , A^k y_r)$ и $A^k y_1, ... , A^k y_r$ - линейно независимые как подсистема базиса $V$ \\
    Тогда по теореме о ранге и дефекте получим, что $defA^k$ в точности равен числу векторов из базиса $V$, лежащих в ядре $A^k$, то есть векторов вида $A^{k_i - l}x_i, 1 \leq l \leq min\{k, k_i\} \Rightarrow N_{\geq k} = defA^k - defA^{k - 1}$
    \item В итоге получаем:
    \begin{gather*}
        N_k = N_{\geq k} - N_{\geq k + 1} = defA^k - defA^{k - 1} - (defA^{k + 1} - defA^k) = \\
        = 2defA^k - defA^{k - 1} - defA^{k + 1}
    \end{gather*}
\end{enumerate}
$\blacksquare$ \\ \\ \\




\newpage
\section*{Расстояния, нормы, скалярные произведения, полиэдры.}
\addcontentsline{toc}{section}{Расстояния, нормы, скалярные произведения, полиэдры.}
\subsection*{Задача 1:}
\addcontentsline{toc}{subsection}{Задача 1:}
\noindent Докажите, что функция $p(x, y) = |x - y|/(1 + |x - y|)$ задает расстояние в вещественном пространстве $\mathbb{R}$. Будет ли пространство полным? \\ \\
\textbf{\underline{Решение:}} \\
Проверим свойства нормы:
\begin{enumerate}
    \item Симметричность - очевидна из симметричности модуля
    $$|x - y|/(1 + |x - y|) = |y - x|/(1 + |y - x|)$$
    \item Неотрицательность - очевидна из неотрицательности числителя и знаменателя $p(x, y) \ge 0$
    \item Неравенство треугольника -
    \begin{gather*}
        p(x, y) = \frac{|x - y|}{1 + |x - y|} = 1 - \frac{1}{1 + |x - y|} \le \text{(неравенство треугольника для модуля)} \le \\
        \le 1 - \frac{1}{1 + |x - z| + |z - y|} = \frac{|x - z| + |z - y|}{1 + |x - z| + |z - y|} = \\
        = \frac{|x - z|}{1 + |x - z| + |z - y|} + \frac{|z - y|}{1 + |x - z| + |z - y|} \le \frac{|x - z|}{1 + |x - z|} + \frac{|z - y|}{1 + |z - y|} = \\
        = p(x, z) + p(z, y)
    \end{gather*}
\end{enumerate}
$\blacksquare$ \\ \\ \\


\subsection*{Задача 2:}
\addcontentsline{toc}{subsection}{Задача 2:}
\noindent Докажите, что множество $S$ замкнуто в метрическом пространстве $M$ тогда и только тогда, когда дополнительное множество $M\backslash S$ открыто. \\ \\
\textbf{\underline{Решение:}} \\
Разобьем решение на два подпункта:
\begin{enumerate}
    \item Покажем достаточность. $S$ - замкнуто, что означает что оно содержит все свои предельные точки. Предположим, что $M\backslash S$ не является открытым, тогда $\exists x \in M\backslash S$, такая что в любой ее окрестности содержится хотя бы одна точка из $S$ (иначе $x$ - внутренняя) $\Rightarrow x$ - предельная для $S$, но $x \not\in S$ - противоречие с замкнутостью $S$. 
    \item Покажем необходимость. $M\backslash S$ - открыто, значит каждая точка этого мн-ва является внутренней. Предположим, что $S$ не замкнуто, тогда $\exists x \in M\backslash S : x$ - предельная для $S \Rightarrow$ в любой окрестности $x$ если точка из $S$ и $x$ не является внутренней - противоречие.
\end{enumerate}
$\blacksquare$ \\ \\ \\


\subsection*{Задача 3:}
\addcontentsline{toc}{subsection}{Задача 3:}
\noindent Пусть точками метрического пространства $M$ являются натуральные числа, а расстояние между $m$ и $n$ определяется как $\rho(m, n) = 1 + min(1/m, 1/n)$ при $m \not= n$ и $0$ при $m = n$. Докажите, что $M$ - полное метрическое пространство. Докажите также, что замкнутые шары $
\overline{M}(1, 1 + 1/2) \supset \overline{M}(1, 1 + 1/3) \supset \overline{M}(1, 1 + 1/4) \supset ...$ вложены, но имеют пустое пересечение. \\ \\
\textbf{\underline{Решение:}} \\
Разобьем решение на два подпункта:
\begin{enumerate}
    \item Покажем достаточность. $S$ - замкнуто, что означает что оно содержит все свои предельные точки. Предположим, что $M\backslash S$ не является открытым, тогда $\exists x \in M\backslash S$, такая что в любой ее окрестности содержится хотя бы одна точка из $S$ (иначе $x$ - внутренняя) $\Rightarrow x$ - предельная для $S$, но $x \not\in S$ - противоречие с замкнутостью $S$. 
    \item Покажем необходимость. $M\backslash S$ - открыто, значит каждая точка этого множ-ва является внутренней. Предположим, что $S$ не замкнуто, тогда $\exists x \in M\backslash S : x$ - предельная для $S \Rightarrow$ в любой окрестности $x$ если точка из $S$ и $x$ не является внутренней - противоречи.
\end{enumerate}
$\blacksquare$ \\ \\ \\



\end{document}
